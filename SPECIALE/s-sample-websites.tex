\section{Sample websites}
\framepage{15cm}{
slashdot.org (examine code - ask developer),  Politiken, DynaWeb (SGI
- oooold check on fyn, when was it first developed?), valueclick
(banners - ask Ask for full story.  What are they running), php3
documentation online.  photo.net (reread).  Sun's docs.sun.com
(DocBook SGML)

Talk about standard search engines on web pages.
}


\subsection{slashdot.org - high volume information site for nerds}
\framepage{15cm}{

Describe setup.  Describe flow, and the slashdot effect.  What
hardware/software.
}

\subsection{Valueclick.com}
\framepage{15cm}{

Banner provider.  Interview Ask for details, about how the NT-cluster
was replaced with a small Linuxbox with MySQL and a squid in
http-accelleration mode.  Explain the problems with having fat
processes serving slow modems.
}
\subsection{Politiken}
\framepage{15cm}{

Talk with Lars.  Politiken genrates their pages on the fly.
}

\subsection{Amazon - an Internet bookstore}
\label{sec:amazon-an-internet-bookstore}

The \myurl{http://www.amazon.co.uk}{Amazon Internet bookstore}
pioneered the virtual Internet bookstore, where the potential buyers
use their browser to see the virtual inventory of literally millions
of books, and order their selections.  These are processed by Amazon
and forwarded to subcontracters who deliver the purchased goods by
mail to the buyers.  The concept has since been copied \textsf{by
  several others}.  

\myimage{gr/amazon-1}{The frontpage of Amazon.co.uk.  An ISBN-number
  has been entered in the very prominent search box}{amazon-1}

The Amazon frontpage is shown in figure~\vref{fig:amazon-1}.  Note the
high amount of potentially interesting links, and the prominently
placed search box.  Due to the limited domain of books, the search
engine can make assumptions about what the user is requesting
information about.  In this case a ten-digit number was entered which
happened to be an ISBN-number, which in most other search engines
should have been entered as ``ISBN-1-55860-5347'', and the page for
the corresponding book was returned (see figure~\vref{fig:amazon-2}).

\myimage{gr/amazon-2}{The result of the search in figure~\vref{fig:amazon-1}}{amazon-2}

This page shows the essential information like author, price, and the
number of pages, but also an image of the cover, the rank on the
Amazon sales list, and a direct link to purchase the book (Amazon
remembers your credit card information, allowing for their patented
``One-click-purchase'' technology).  What makes the Amazon web page
truly better that a printed catalogue is that the users are encouraged
to submit feedback, and the information Amazon collects from online
sales.  These features are important for Amazon:

\begin{itemize}
\item \textbf{Average customer rating} - this is the essence of all
  the reviewers opinions, on a scale from 0 to 5.
\item \textbf{Reviews} - readers take the time to create comprehensive
  reviews of the books, all of which Amazon then places prominently on
  the page, with due credit to the individual author.   Recently a
  ``Was this review helpful to you?'' button was added to each review,
  and the result of each poll is shown along with the review, giving
  the content more value.
\item \textbf{Similar books} - if a previous buyer bought other books
  along with this one, they might be interesting to this person too.
  Links are presented to those books, as well as to the general
  categories this book was placed in by the Amazon librarian.

\item \textbf{Expected content} - Amazon can be expected to have
  \textit{any} English book in their database.  Users expect to be
  able to find a given book in the Amazon database, and usually do.
  Personally I have never looked in vain at Amazon for English books.
\end{itemize}

Amazon explicitely invites those who read the page to review it, with
special treatment for the author and the publisher.  \textsf{As
  Greenspun observes, then this is the most efficient material}.  At
the time of writing, this book has 9 reviews at \textsf{amazon.co.uk},
but 198 at the mother site at \textsf{amazon.com}.  The corresponding
HTML-pages are 22 kb and 550 kb respectively, which gives an an impressive
\textsf{Greenspun factor} (setting the UK-version to be 100\%
\textsf{??-provider}) of 25.

Conclusion:  The Amazon family of sites are very good, and do what
they can to improve with the help of visitors.  Amazon have been able
to keep their leader status in the virtual bookshop niche, by provide excellent


% \subsection{The United States Library of Congress}
% \label{sec:congress}

% \textsf{CHANGED TO CATALOG.LOC.GOV - much improved...}

% The \myurl{http://www.loc.gov}{United States Library of Congress} is
% the definite place to search if you actually need a given physical
% book.  Unfortunately their website does not help much getting it.  The
% front page have a ``Search'' links to
% \myurl{http://www.loc.gov/catalog/}{a page with a choice} between 
% ``Simple Search'', and ``Advanced Search''.
% Figure~\vref{fig:congress-1} shows the ``Simple Search'' page with the
% title ``art of computer programming'' entered, wishing to get
% references to the Donald Knuth tomes.  

% \myimage{gr/congress-1}{The Library of Congress ``Simple Search''
%   page}{congress-1}

% The corresponding ``Advanced Search'' page is shown as
% figure~\vref{fig:congress-2} (\textsf{compare to google} and its
% simple syntax).  The close correspondence between the form and the
% resulting SQL-statement, makes it difficult to use for most users.

% \myimage{gr/congress-2}{The Library of Congress ``Advanced Search''
%   page}{congress-2}

% The result of the search is shown as figure~\vref{fig:congress-3}.  It
% is very clear that the web interface calls an underlying search
% engine, and basically submits the textual results back as a web page
% (with a few navigational links added).

% \myimage{gr/congress-3}{The result of the search in
%   figure~\vref{fig:congress-1}}{congress-3}

% \myimage{gr/congress-4}{``More on this record'' for record 1 in figure
%   ~\vref{fig:congress-3}}{congress-4}

% Figure~\vref{fig:congress-4} shows the expanded view of the first
% record.  The web page is again pure text with a few annotations which
% deals with presentation only, meaning that this is web-wise a dead
% end.  The similarity to the original Odin-WWW interface written around
% 1996 is stunning (Odin was upgraded December 1999 so a screen shot
% cannot be shown), and is a clear indication of how fast the
% expectations regarding the facilities of a given website have changed.

% Conclusion:  This is clearly not a top priority for the Library of Congress


\subsection{www.krak.dk}
\label{sec:www.krak.dk}

The Danish map manufacturer \myurl{http://www.krak.dk}{Krak A/S} has
a reputation of being the definite guide on maps, especially on
driving maps in Copenhagen.  Their website have been up for about two
years now, and have been steadily improved with new facilities and
cross-references.  The start page is shown in figure~\vref{fig:krak-1},
and is seperated in two parts; namely white pages (phone number
based), and pink pages (company based).

\myimage{gr/krak-1}{Welcome page (and search form) for www.krak.dk}{krak-1}

The fields in the white page section are Name, Road, House number, Zip
code, City and Phonenumber.  The fields in the pink page section are
Company and Phonenumber.  The form has been filled out with a search request
for ``Odense Universitet'', and the results are shown in
figure~\vref{fig:krak-2}. 

\myimage{gr/krak-2}{Result of searching for ``Odense University''}{krak-2}

Each line containing an answer may have icons referencing
to facilities regarding the location of the answer.  These answers
have pointers to 
Rejseplanen, the Route Planner, and a map reference.  Additionally
pointers to a web page, and an email address could have been provided
by the users.  Clicking the map icon of the first line actually
referring to Odense University brings us to figure~\vref{fig:krak-3}.

\myimage{gr/krak-3}{Following the map icon for the first ``Odense
  University'' reference}{krak-3}

The blue dot usually indicates the location of the address in
question, but in this case it is a bit off (the University is located
500 meters further to the south).  When a map is display, the
navigation area below the image allows for zooming and moving the
contents of the shown area.  By selecting ``Zoom out'' and ``x8'' a
new map is shown (see figure~\vref{fig:krak-4}), where it is evident
that the database show a great deal of detail.  Roads, residential
areas, streams, train stations, the motor way and green areas are
shown.  These maps provide a level of information corresponding to
Krak's printed maps.

\myimage{gr/krak-4}{The map in figure~\vref{fig:krak-3} zoomed out
  with a factor 8}{krak-4}

Now go back to the list of results from the search for ``Odense
University''.  The Car-icon gives a route to the listed location,
where you must fill in the source yourself.  Figure~\vref{fig:krak-5a}
show the form filled in with my own address and the address of the
university, and ``Route on map'' (Rute p� kort) gives a visual route
between the two locations.

\myimage{gr/krak-5a}{???????????????????????
  University'' reference}{krak-5a}

This map is shown in figure~\vref{fig:krak-5b}, and is perfectly
adequate for a person driving a car.  For cyclists it is often a good
idea to look for short-cuts, if you are well known in the area.

\myimage{gr/krak-5b}{???????????????????????
  University'' reference}{krak-5b}

If you need step-by-step directions, Krak can provide that too.
Select the ``\textsf{???}'' button and print out the directions.

\myimage{gr/krak-6}{???????????????????????
  University'' reference}{krak-6}


\framepage{15cm}{ Necessary to do rendering according to
  requests.  Cannot be predicted reasonably.  What have been done?
  How does it work?  What about }


\subsection{The Journay Planner - www.rejseplanen.dk}
\label{sec:www.rejseplanen.dk}

DSB (Danish Rail) have a journey planner which is based on train
stations, and major bus stops.  Figure~\vref{fig:dsb-1} shows the initial
form where the two end points for the journey as well as the arrival
or departure time is indicated.  Please note that the two surrounding
frames (left side and top - the scroll bare on the right indicates the
actual area available) leave only \textsf{70\% of the area} to the
application itself.

\myimage{gr/dsb-1}{Rejseplanen1}{dsb-1}

\textsf{Figure~\vref{fig:dsb-2} is the same as the previous figure}???

\myimage{gr/dsb-2}{Rejseplanen2}{dsb-2}

The search returns a number of potential journeys, with departure and
arrival times, total expected travel time, and the number of changes
necessary during the journey.  The departure corresponding the best
for the indicated period is highlighted with the blue bar.

\myimage{gr/dsb-3}{Rejseplanen3}{dsb-3}

Figure~\vref{fig:dsb-3}

\myimage{gr/dsb-4}{Rejseplanen4}{dsb-4}
Figure~\vref{fig:dsb-4}
\myimage{gr/dsb-5}{Rejseplanen5}{dsb-5}
Figure~\vref{fig:dsb-5}


\subsection{Freshmeat.net}
\label{sec:freshmeat.net}

\subsection{Google/Altavista}

\framepage{15cm}{
  Search engines.  How do they do it?  www.909.dk, www.krak.dk,
  terraserver.microsoft.com.  Encyclopedia Brittanica.

  ASP+Access, tinderbox \& bugzilla, javasoft (developers area),
  LXR+Bonsai.  ``Alle danskere paa nettet'' - Dansk Journalistforbund,
  Pressemeddelelse. Greenspun selv?
  }

\subsection{The Collection of Computer Science Bibliographies}
\myurl{http://liinwww.ira.uka.de/bibliography/index.html}{http://liinwww.ira.uka.de/bibliography/index.html}


\subsection{Jydske Bank}
\label{sec:jydske-bank}

THe only java-solution for home banking.  Any comments?

\subsection{Cactus -- document capture and conversion}

\framepage{15cm}{
Two out of four methods are implemented (email, printer).  (Increase
blob limit in DBI::MySQL).  Sample PDF viewer, perhaps simple XLS viewer.

Demonstrate that Cocoon can provide the navigational framework.
}

%%% Local Variables: 
%%% mode: latex
%%% TeX-master: "rapport"
%%% End: 

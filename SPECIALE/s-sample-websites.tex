\chapter{Sample websites and projects}

% \framepage{15cm}{
% slashdot.org (examine code - ask developer),  Politiken, DynaWeb (SGI
% - oooold check on fyn, when was it first developed?), valueclick
% (banners - ask Ask for full story.  What are they running), php3
% documentation online.  photo.net (reread).  Sun's docs.sun.com
% (DocBook SGML)
%
% Talk about standard search engines on web pages.
%
% http://www.useit.com/alertbox/991017.html
% }

%\textsf{Beskriv projekt fra Jyllandspostudklip}

\section{slashdot.org - high volume information site for nerds}

\myimage{gr/slashdot-1}{The Slashdot start page}{slashdot-1}

\myurl{http://slashdot.org}{SlashDot} was started in September 1997
with the slogan ``News for nerds - Stuff that matters'', and do so by
providing dynamically generated web pages with a lot of articles
categorized with an icon, where a short summary is shown along with
links to the full story, several references to the orignal sources,
the authors email address, plus an overview of the number of comments
in the discussion forum.  See \myvref{fig:slashdot-1} for the top
of the start page at March 14 2000.

\myimage{gr/slashdot-2}{A Slashdot article with the top of the comments}{slashdot-2}

When the ``Read more'' link to a given article is followed, a page is
shown with the full text of the posting, followed by a long list of
comments submitted by readers (see \myvref{fig:slashdot-2}), and
a login box.

Unregistered readers are allowed to post under the generic name of
``Anonymous Coward'', where registration allows readers to select
their own handle, full name, home page link, page layout and what kind
of articles they want to see.

\myimage{gr/slashdot-3}{My personal page at  Slashdot}{slashdot-3}

After logging in, the ``Welcome Back'' screen in
\myvref{fig:slashdot-3} is shown.  Note that the layout has changed,
and that I have not moderated anything.  Slashdot uses moderation
where everybody can grade other posters comments which is then
available to users for their selection of articles.  Personally I have
chosen a minimum level of two, meaning that I never see anything from
Anonymous Cowards (which improves the quality immensely).  Meta
moderation is intended to control the quality of the moderation, and
is a volunteer action.

% Customization... Left out
%\myimage{gr/slashdot-4}{The Slashdot start page 4}{slashdot-4}


% \myvref{fig:slashdot-4}

% \myimage{gr/slashdot-5}{The Slashdot start page 5}{slashdot-5}
% \myvref{fig:slashdot-5}

\begin{figure}[htbp]
  \begin{center}
\begin{alltt}
Slashdot Stats

date: 2:37am 
uptime: 68 days, 5:56, 2 users, 
load average: 0.18, 0.19, 0.18 
processes: 65 
yesterday: 209994
today: 1
ever: 278137148
\end{alltt}
\caption{The Slashdot statistics for a day in March 2000}
    \label{fig:slashdot-stats}
  \end{center}
\end{figure}

Slashdot provide the ``magnet content'' discussed by Greenspun
\cite{phillipandalexsguidetowebpublishing}.  Stories are carefully
selected, and the user may even select just a few categories in her
personal preferences.

\textit{All} pages are dynamically generated --
\myvref{fig:slashdot-stats} lists this to be around 200,000 pages per
day.  The discussion forums for any story is active for a reasonable
time, where anybody can comment on any other comment, and then
``frozen'' when the story is archived.  If a user at a later date
wishes to see the story with the discussion, it is retreived from the
backup database.

The ``Ask Slashdot'' feature is a direct consequence of the fact that
Slashdot is targeting power computer users.  By allowing carefully
selected questions to a highly visible spot, the original inquirer is
usually able to get a very qualified answer.  The question along with
all the answers are stored in the same way as a story.

Due to the prominent position of new postings up front, many readers
follow the given links causing the so-called \texttt{Slashdot effect}
where a webserver suddenly gets a lot of simultaneous requests, which
may overload it so much that it cannot serve all of the incoming
requests, returning error messages.  The Slashdot Effect is
\myurl{http://ssadler.phy.bnl.gov/adler/SDE/SlashDotEffect.html}{discussed
in an Internet paper by Stephen Adler}.  \myvref{fig:slashdoteffect-1}
is taken from this paper, and shows the number of requests pr second
for the slashdotted webserver according to its own log.

\myimage{gr/fiw98Zoom}{A webserver log showing the Slashdot effect --
  the red line at approximately 9088.9 shows when the announcement was
  made on Slashdot}{slashdoteffect-1}

\section{Valueclick.com}
  
ValueClick is one of several banner-ad sellers on the Internet.
Banner ads are animated images which advertise for a given website,
and which link to the website in question.  The advertiser pay
ValueClick for a certain amount of ``\textit{click-throughs}'' (users
who actually click the image to go to the advertised site) as opposed
to the traditional number of exposures.  The former is reputedly
harder to track than the latter.

ValueClick claims on their web site that they send out 40 million banner
ads every day -- that is 460 ads a second -- so it is vital for
business that the counters for each advertiser are accurately updated.
From private email I have learned that their original NT-cluster based
solutions were not good enough for several reasons, most notably the
webservers having to wait for the browsers to accept every byte sent
to them, before they could service the next user request.  Proper
handling of this and other similar problems with OpenSource software
allowed a single Linux based Pentium machine to outperform the rest
of the server park.

Today ValueClick is based on a large number of FreeBSD machines which
use MySQL as their database server, and apparently works very well.



% \section{Politiken}
% \framepage{15cm}{

% Talk with Lars.  Politiken genrates their pages on the fly.
% }

\section{Amazon - an Internet bookstore}
\label{sec:amazon-an-internet-bookstore}

The \myurl{http://www.amazon.co.uk}{Amazon Internet bookstore}
pioneered the virtual Internet bookstore, where the potential buyers
use their browser to see the virtual inventory of literally millions
of books, and order their selections.  These are processed by Amazon
and forwarded to subcontracters who deliver the purchased goods by
mail to the buyers.  The concept has since been copied by several
other bookstores.

\myimage{gr/amazon-1}{The frontpage of Amazon.co.uk.  An ISBN-number
  has been entered in the very prominent search box}{amazon-1}

The Amazon frontpage is shown in \myvref{fig:amazon-1}.  Note the
high amount of potentially interesting links, and the prominently
placed search box.  Due to the limited domain of books, the search
engine can make assumptions about what the user is requesting
information about.  In this case a ten-digit number was entered which
happened to be an ISBN-number, which in most other search engines
should have been entered as ``ISBN 1-55860-5347'', and the page for
the corresponding book was returned (see \myvref{fig:amazon-2}).

\myimage{gr/amazon-2}{The result of the search in \myvref{fig:amazon-1}}{amazon-2}

This page shows the essential information like author, price, and the
number of pages, and an image of the cover, the rank on the
Amazon sales list, and a direct link to purchase the book (Amazon
remembers your credit card information, allowing for their patented
``One-click-purchase'' technology).  What makes the Amazon web page
truly better that a printed catalogue is that the users are encouraged
to submit feedback, and the information Amazon collects from online
sales to provide information to potential customers about other books
that might interest them, because they interested other customers
which bought the book they are looking at now.

These features are distinct for a page for a purchasable item at
Amazon:

\begin{itemize}
\item \textbf{Average customer rating} -- this is the essence of all
  the reviewers opinions, on a scale from 0 to 5.
\item \textbf{Reviews} -- readers take the time to create comprehensive
  reviews of the books, all of which Amazon then places prominently on
  the page, with due credit to the individual author.   Recently a
  ``Was this review helpful to you?'' button was added to each review,
  and the result of each poll is shown along with the review, allowing
  a customer to rapidly determine how to rate the review.

\item \textbf{Similar books} -- if a previous buyer bought other books
  along with this one, they might be interesting to this person too.
  Links are presented to those books, as well as to the general
  categories this book was placed in by the Amazon librarian.
  
\item \textbf{Expected content} -- Amazon have many books in their
  database, and any recently published book can be expected to be
  found.  Users return with an expectation that their searches will
  succeed.  John Immerk{\ae}r reports, however, that Amazon
  occasionally do not have all the books he need in their database.

\end{itemize}

Amazon explicitly invites those who read the page to review the book
if they have read it, with special treatment for the author and the
publisher.  As Greenspun observes
in~\cite{phillipandalexsguidetowebpublishing} this is a very efficient
method of getting information of a very high quality, and that a
reasonable way for him to measure the success of a page, is to compare
the size of the original content to the size of the comments provided
by users.  At the time of writing, this book has 9 reviews at
\texttt{amazon.co.uk}, but 198 at the mother site at
\texttt{amazon.com}.  The corresponding HTML-pages are 22 kb (a good
approximation of the page without comments) and 550 kb respectively.
That is 24 times as many comments as original text.  Not bad.

Amazon uses very aggressive marketing, which has caused it to be a
well-known brand name in just a few years.  By giving money to those
who link to their pages, as well as pay search engines to have links
to the Amazon site show up at the top in just about every search, they
have managed to be just about the only Internet bookstore known to
most people.  

% Conclusion: The Amazon  family of sites are very good, and do what they
% can to improve with the help of visitors.  Amazon have been able to
% keep their leader status in the virtual bookshop niche, by living up
% to the customers expectations.  Hopefully, they will make money one
% day.



\section{Bang and Olufsen}
\label{sec:bang-and-olufsen}

\myimage{gr/b-and-o-1}{The Bang and Olufsen pages -- notice the three
  randomly selected images at the top}{b-and-o-1} The
\myurl{http://www.bang-olufsen.com/}{Bang and Olufsen} web-site was
previously very unusual in the way that they offered random
navigation.  Their front page contained very few links, and some
product images which could be clicked and lead to a large set of pages
(see \myvref{fig:b-and-o-1}, with three random clickable images
at the top.  These were chosen randomly from the large set of such
pages (600 or more) and put at the top.  This has been highly
efficient in keeping people wandering around and reading their pages,
giving these people an unusual experience.

The underlying database is used for storing the many pages, as well as
for tracking the users as they roam the site.


\section{Deja - where did Usenet go?}
\label{sec:deja}

Deja (originally DejaNews) has evolved around the
\myurl{http://www.deja.com/usenet}{Deja Usenet archive}, where all
Usenet postings since March 1995 has been archived (accounting for 500
gigabytes of disk space) and can be searched in.

In addition to the normal search for a given message, the search
possibilities of Deja can also produce complete profiles of a given
author based on email address or full name, allowing anyone to get a
good idea of your personal interests on Usenet.

Due to its long-term storage facility, Deja is a good place to go if
your local news-server has expired a given posting.



\section{www.krak.dk}
\label{sec:www.krak.dk}

Making maps on demand was one of the earliest applications for the web
as mentioned in
\myvref{sec:map-generation-first-web-application}.  In Denmark,
the well renowned Danish map manufacturer
\myurl{http://www.krak.dk}{Krak A/S} opened their Internet phone
directory and web map generation service in direct competition with
numerous other directory services.  Their advantage was up-to-date
data from the databases of Tele Danmark combined with an ability to
generate a map for a given address, whether it was entered directly or
being a byproduct of a search for something else.  This has proven to
be so powerful a combination of services that all other phone
directory services has been left far behind.

The website has been steadily improved with new facilities and
cross-references, as well as more precise data (In 1999 it could not
locate ``Herlev Hovedgade 205'' on the map, so the whole of the road
was inked on the map.  This is corrected today).  The start page is
shown in \myvref{fig:krak-1}, and is seperated in two parts;
namely white pages (phone number based), and pink pages (company
based).

\myimage{gr/krak-1}{Welcome page (and search form) for www.krak.dk}{krak-1}

The fields in the white page section are Name, Road, House number, Zip
code, City and Phone number.  The fields in the pink page section are
Company and Phone number.  The form has been filled out with a search
request for ``Odense Universitet'', and the results are shown in
\myvref{fig:krak-2}.

\myimage{gr/krak-2}{Result of searching for ``Odense University''}{krak-2}

Each line containing an answer may have icons referencing to
facilities regarding the location of the answer.  These answers have
pointers to Rejseplanen, the Route Planner, and a map reference.
Additionally pointers to a web page, and an email address could have
been provided by the users.  Clicking the map icon of the first line
actually referring to Odense University brings us to
\myvref{fig:krak-3}.

\myimage{gr/krak-3}{Following the map icon for the first ``Odense
  University'' reference}{krak-3}

The blue dot usually indicates the location of the address but in this
case the map is not 100\% correct (Moseskovvejen does not go all the
way south-east to the parking lot\footnote{John Immerk{\ae}r has a
  theory that this is caused by the limited resolution of the database
  used to draw the maps.  The forest is very rough in the edges, and
  no bicycle paths are shown. This is most likely done to make the map
  drawing faster in the server.  The abrupt ending of Moseskovvejen is
  most likely a result of the bicycle path ending there, and the total
  width of the road being too narrow to be included in the map
  database} -- and the University is located 500 meters further to the
south).  When a map is display, the navigation area below the image
allows for zooming and moving the contents of the shown area.  By
selecting ``Zoom out'' and ``x8'' and clicking on the blue dot a new
map is shown (see \myvref{fig:krak-4}), where it is evident that
the database show a great deal of detail.  Roads, residential areas,
streams, train stations, the motor way and green areas are shown.
These maps provide a level of information corresponding to Krak's
printed maps.

\myimage{gr/krak-4}{The map in \myvref{fig:krak-3} zoomed out
  with a factor 8}{krak-4}

Now go back to the list of results from the search for ``Odense
University''.  The Car-icon gives a route to the listed location,
where you must fill in the source yourself.  \myvref{fig:krak-5a}
show the form filled in with my own address and the address of the
university, and ``Route on map'' (Rute p� kort) gives a visual route
between the two locations, along with an estimate of the distance and
time it will take.

\myimage{gr/krak-5a}{The entry box where the starting and ending point
  is entered}{krak-5a}

This map is shown in figure~\myvref{fig:krak-5b} (and explained in words
in \myvref{fig:krak-6}), and is perfectly adequate for a person
driving a car.  For cyclists it is often a good idea to look for
short-cuts, if you are well known in the area.  

\myimage{gr/krak-5b}{The route from my home to the University}{krak-5b}

\myimage{gr/krak-6}{The route in \myvref{fig:krak-5b} explained
  in words}{krak-6}

If you need step-by-step directions, Krak can provide that too.
Select the tiny car button next to the entry and print out the
directions.  Careful studies of these directions can give an idea of
the network grid that the route planner use for finding the shortest
road.

%\textsf{what is~\myvref{fig:krak-6}?}


\section{The Journey Planner - www.rejseplanen.dk}
\label{sec:www.rejseplanen.dk}

DSB (Danish Rail) have a journey planner which is based on train
stations, and major bus stops.  \myvref{fig:dsb-1} shows the initial
form where the two end points for the journey as well as the arrival
or departure time is indicated.  Please note that the two surrounding
frames (left side and top - the scroll bare on the right indicates the
actual area available) leave only about 70\% of the area to the
application itself.

\myimage{gr/dsb-1}{Rejseplanen -- welcome page}{dsb-1}
%\myimage{gr/dsb-2}{Rejseplanen \textsf{??}}{dsb-2}
\myimage{gr/dsb-3}{Results for a search for a connection between
  Odense and {\AA}rhus}{dsb-3}
\myimage{gr/dsb-4}{More details about one of the possible journeys found}{dsb-4}
\myimage{gr/dsb-5}{Results for a search for a connection between
  Esbjerg and Malmparken}{dsb-5}

The search returns a number of potential journeys shown in
\myvref{fig:dsb-3}, with departure and arrival times, total
expected travel time, and the number of changes necessary during the
journey.  The departure corresponding the best for the indicated
period is highlighted with the blue bar.  By accepting the default
selection in the blue bar, and scrolling down to press a button,
\myvref{fig:dsb-4} is shown.  This is a direct connection, and
it is possible to reserve a seat directly from this screen.

A more complicated request is shown in \myvref{fig:dsb-5} where a
connection from Malmparken (suburb to Copenhagen) to Esbjerg (Western
Jutland) is requested.  A route involving three trains and a bus is
suggested, which is very reasonable and probably the fastest too.

Rejseplanen is helped by the fact that the number of nodes in the grid
can be small compared to the journey planner at Krak
, but the
presentation is not good enough.  It is generally too hard to get to
the information if you need a slightly alternate view from what the
programmers expected.

I used to live next to Odense Sygehus, which is a stop on the
Odense-Svendborg railroad, but where not all trains stop.  The
question regarding whether it would be faster to walk to the Odense
Sygehus station or take my bicycle to Odense Station was very hard to
answer by Rejseplanen, since it does not just simply allow access to
the underlying database.  I ended up calling the station and asking
them to look in their paper copy.




% \section{Freshmeat.net}
% \label{sec:freshmeat.net}
%
% Freshmeat.net is a service for locating software.  \textsf{a lot of
%   products.  users can announce software, and point people toward
%   homepages and download pages.  Very nice}
%
%
% \framepage{15cm}{
% \textsf{Check if there still is companies which boost your presence on
%   web search engines}
%
% \textsf{Businesses - notably in porno - discovered this and polluted
%   their web pages with keywords to boost their chance of being
%   returned as a result of a search.}
%
% \textsf{Look for statistics on search engines and ``sex'' - talk about
%   that these businesses have a great interest in attracting you since
%   they cannot rely on you seeing them elsewhere (like usenet spam)}
%
% \textsf{look for email about somebody who tried google and was pleased
%   after being driven away from search engines.   Why does google
%   tick?}
% }

\section{Searching for images - \texttt{ftpsearch.lycos.com}}
\label{sec:searching-for-images-ftpsearch-lycos-com}

\myimage{gr/ftp-search-1}{The Lycos FTP search engine}{ftp-search-1}

Many sites provide a ``Search the Internet'' facility, and they
usually gather the underlying database by using \textit{spiders} to
examine, retrieve and index all the web-pages they can.

A few allow more than that.  An interesting by-product of the basic
web indexing scheme is the Lycos FTP-search facility at
\myurl{http://ftpsearch.lycos.com}{http://ftpsearch.lycos.com} (see
\myvref{fig:ftp-search-1}, which among many things allow searching for
\textit{images} based on the filenames of the images found in web
pages along with the surrounding text, and keywords embedded in the
images.

This works quite well -- most of the cactii images for my project was
found this way.


% \framepage{15cm}{
% \label{sec:google}
%   Search engines.  How do they do it?  www.909.dk, www.krak.dk,
%   terraserver.microsoft.com.  Encyclopedia Brittanica.
%
%   ASP+Access, tinderbox \& bugzilla, javasoft (developers area),
%   LXR+Bonsai.  ``Alle danskere paa nettet'' - Dansk Journalistforbund,
%   Pressemeddelelse. Greenspun selv?
%   }

% \section{The Collection of Computer Science Bibliographies}
% \myurl{http://liinwww.ira.uka.de/bibliography/index.html}{http://liinwww.ira.uka.de/bibliography/index.html}
%
% \textsf{screen shot}
%
% This is a great web-site for scientists because in addition to the
% basic search facilities it allows you to download the BibTeX entry for
% the books and articles you find.

\section{Home banking -- sample bank: F{\ae}llesbanken}
\label{sec:faellesbanken}

% NB...  fejlstavning i filnavn...
\myimage{gr/faelleskasssen-1}{Web banking in F{\ae}llesbanken}{faelleskassen-1}

\myurl{http://www.w57.dk}{Jan Bertelsen} reports that the Home Banking
facility provided by the Danish bank F{\ae}llesbanken is Java based,
allowing him to use it with his Linux computer at home (see
\myvref{fig:faelleskassen-1}).  His experience is basically that it
works well but is slow, especially when cryptography is used.

The reported slowness is most likely due to him using the stock
Netscape browser Java implementation, which is infamous for its
slowness.

Many people have expressed that they wanted Java based Home Banking
since that allowed them to choose their operating system freely,
instead of being forced to use a Microsoft product.   Several
OS/2-users chose bank based on whether their Home Banking solution
could run under OS/2 or OS/2's Java.  


% \section{3coms elektroniske papir}

% \textsf{Beskriv projekt fra Jyllandspostudklip}


\section{Yggdrasil - a simple navigational framework}
\label{sec:yggdrasil}

Based on an idea by Anders Stengaard S{\o}rensen, I designed and
implemented the Yggdrasil system to improve the use of the Intranet at
AMROSE A/S.  The basic idea was:

\begin{itemize}
\item Everybody should be allowed to publish document to the internal
  webserver.  A string in the title of each document decides where it
  should go
\item The navigational framework supporting user navigation between
  the various documents, should be generated by a program 
\end{itemize}

The following quote is from the original announcement:

\begin{quotation}
  A need for an automatic maintainance of the central webpages was
found during the construction of the intranet documentation in AMROSE
A/S in 1997. These webpages primarily contain pointers to user written
pages, as opposed to the more traditional approach of having a
dedicated webmaster maintaining the various pages.

In AMROSE the need was more for the individual users to easily make
additions to a common pool of documents, without having to appoint a
webmaster. The Yggdrasil principle is to let the users put markers in
the individual documents which identifies the placement of the link in
the central webpages, and letting them maintain this information in
the true spirit of the web, i.e. decentralised. Additionally, this is
unrelated to any physical location of the given document, enabling
users to provide information without having to obey rules regarding
placements of physical files. Actually, a planned enhancement of
Yggdrasil will allow the documents to span several webservers.

Yggdrasil provides several views on the document database. The primary
one is the one where the user navigates the hierachy of the document
categories , and where only the documents appropriate to your current
selection are displayed. This goes very well together with the
overview where documents are sorted according to their revision date.
It is very easy to see which documents have been added since a given
date.
\end{quotation}

\myimage{gr/yggdrasil-1}{Typical Yggdrasil navigational
  screen}{yggdrasil-1}  


Several prototypes were constructed, and the version in use when I
left AMROSE, scanned all the individual web pages on the system,
looked for the magic ``[Class1/class2...]'' string in the title, and
created a large set of pages implementing

\begin{itemize}
  
\item A list of documents per user.
  
\item A ``Recently Changed Pages'' list of documents sorted after
  entry date.  This allowed for catching up quickly for a given
  period.
  
\item A set of right and left side panes for the navigation.  Clicking
  a link (like those shown in \myvref{fig:yggdrasil-1}) changed
  the contents to another page where the wanted action had been taken.
  
\end{itemize}

The users were expected to

\begin{enumerate}
\item Write a document with the editor in Netscape Navigator
\item Set the classification string in the title
\item Upload it to their ftp directory on the server
\item Send an email to a special Yggdrasil user, which would trigger
  the rebuild of the pages (in addition to the nightly rebuild).  The
  update would normally take less than 30 seconds.
\end{enumerate}

Yggdrasil in itself consisted of about 10 pages of perl code.  The
search facility was provided by an \myurl{http://htdig.org}{ht://dig
installation}, which rebuilt its database every night.


This worked very well in the beginning but had a fatal blow when --
due to the price of Netscape Navigator -- it was decided that the
company was to use Microsoft Internet Explorer only.  The users were
to edit their documents in Microsoft Word, which had very poor support
for exporting to HTML and the facilities for uploading documents to
the webserver was also rather tedious to use.  Users of Word could not
enter the information fields recognized by Yggdrasil, making it even
harder to publish documents.

Users started publishing raw Word documents with small HTML wrappers,
and the usage quickly dwindled since it was too cumbersome to use.

Unfortunately, the management never fully backed the Yggdrasil
project, new users were not informed about the system and required to
use it, meaning that these problems were not resolved and after a time
of neglect the system was no longer in use.

The Cactus system is my response to the experience learned from this
project, as discussed in \myvref{cha:cactus}.


\section{The TOM browser - converting documents online}
\label{sec:tom}

\myimage{gr/tom-server-1}{The TOM Conversion server}{tom-server-1}

\myurl{http://wheel.compose.cs.cmu.edu:8001/cgi-bin/browse}{The TOM
  browser} is a document conversion system which allows users to
upload files and provide URL's to document on the web, which is then
converted to one of numerous formats.  The default is to autodetect
the file type, and convert to a viewable format.

I discovered this site so late that I did not have time to investigate
it fully, but I find it so interesting that I wanted it to be listed
here.  It is similar to the Cactus idea but on a much narrower basis,
as it is a single item conversion scheme.



% \section{Cactus -- document capture and conversion}

% \framepage{15cm}{
% Two out of four methods are implemented (email, printer).  (Increase
% blob limit in DBI::MySQL).  Sample PDF viewer, perhaps simple XLS viewer.

% Demonstrate that Cocoon can provide the navigational framework.

% A full description is available in~\section{site:cactus}
% }
%%% Local Variables: 
%%% mode: latex
%%% TeX-master: "rapport"
%%% End: 

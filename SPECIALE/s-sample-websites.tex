\chapter{Sample websites and projects}
\framepage{15cm}{
slashdot.org (examine code - ask developer),  Politiken, DynaWeb (SGI
- oooold check on fyn, when was it first developed?), valueclick
(banners - ask Ask for full story.  What are they running), php3
documentation online.  photo.net (reread).  Sun's docs.sun.com
(DocBook SGML)

Talk about standard search engines on web pages.

http://www.useit.com/alertbox/991017.html
}


\section{slashdot.org - high volume information site for nerds}

\myimage{gr/slashdot-1}{The Slashdot start page}{slashdot-1}

\myurl{http://slashdot.org}{SlashDot} was started in September 1997
with the slogan ``News for nerds - Stuff that matters'', and do so by
providing dynamically generated web pages with a lot of articles
categorized with an icon, where a short summary is shown along with
links to the full story, several references to the orignal sources,
the authors email address, plus an overview of the number of comments
in the discussion forum.  See figure~\vref{fig:slashdot-1} for the top
of the start page at March 14 2000.

\myimage{gr/slashdot-2}{A Slashdot article with the top of the comments}{slashdot-2}

When the ``Read more'' link to a given article is followed, a page is
shown with the full text of the posting, followed by a long list of
comments submitted by readers (see figure~\vref{fig:slashdot-2}), and
a login box.

Unregistered readers are allowed to post under the generic name of
``Anonymous Coward'', where registration allows readers to select
their own handle, full name, home page link, page layout and what kind
of articles they want to see.

\myimage{gr/slashdot-3}{My personal page at  Slashdot}{slashdot-3}

After logging in, the ``Welcome Back'' screen in
figure~\vref{fig:slashdot-3} is shown.  Note that the layout has
changed, and that I have not moderated anything.  Slashdot uses
moderation where everybody can grade other posters comments which is
then available to users for their selection of articles.  Personally I
have chosen a minimum level of two, meaning that I never see anything
from Anonymous Cowards (which improves the quality immensely).  Meta
moderation is intended to control the quality of the moderation, and
is a volunteer action.

% Customization... Left out
%\myimage{gr/slashdot-4}{The Slashdot start page 4}{slashdot-4}


% \vref{fig:slashdot-4}

% \myimage{gr/slashdot-5}{The Slashdot start page 5}{slashdot-5}
% \vref{fig:slashdot-5}


\begin{verbatim}
Slashdot Stats

date: 2:37am 
uptime: 68 days, 5:56, 2 users, 
load average: 0.18, 0.19, 0.18 
processes: 65 
yesterday: 209994
today: 1
ever: 278137148
\end{verbatim}

Slashdot provide the ``magnet content'' Greenspun talks about.
Stories are carefully selected, and the user may even select just a
few categories in her personal preferences.

\textit{All} pages are dynamically generated.  The discussion forums for
any story is active for a reasonable time, where anybody can comment
on any other comment, and then ``frozen'' when
the story is archived.  If a user at a later date wishes to see the
story with the discussion, it is retreived from the backup database.

The ``Ask Slashdot'' feature is a direct consequence of the fact that
Slashdot is targeting power computer users.  By allowing carefully
selected questions to a highly visible spot, the original inquirer is
usually able to get a very qualified answer.  The question along with
all the answers are stored in the same way as a story.

Due to the prominent position of new postings up front, many readers
follow the given links causing the so-called \texttt{Slashdot effect}


\myimage{gr/fiw98Zoom}{A webserver log showing the Slashdot effect --
  the red line at approximately 9050 shows when the announcement was
  made on Slashdot}{slashdoteffect-1}

\section{Valueclick.com}
  
ValueClick is one of several banner-ad sellers on the Internet.
Banner ads are animated images which advertise for a given website.
The advertiser pay ValueClick for a certain amount of
``\textit{click-throughs}'' (users who actually click the image to go
to the advertised site) as opposed to the traditional number of
exposures.  The former is harder to track than the latter.

ValueClick claims on their web that they send out 40 million banner
ads every day -- that is 460 ads a second -- so it is vital for
business that the counters for each advertiser is accurately updated.
From private email I have learned that their original NT-cluster based
solutions were not good enough for several reasons, most notably the
webservers having to wait for the browsers to accept every byte sent
to them, before they could service the next.  By handling this and
other similar problems, a single Linux based Pentium machine could
outperform the rest of the server park.

Today ValueClick is based on a large number of FreeBSD machines which
use MySQL as their database server, and apparently works very well.



% \section{Politiken}
% \framepage{15cm}{

% Talk with Lars.  Politiken genrates their pages on the fly.
% }

\section{Amazon - an Internet bookstore}
\label{sec:amazon-an-internet-bookstore}

The \myurl{http://www.amazon.co.uk}{Amazon Internet bookstore}
pioneered the virtual Internet bookstore, where the potential buyers
use their browser to see the virtual inventory of literally millions
of books, and order their selections.  These are processed by Amazon
and forwarded to subcontracters who deliver the purchased goods by
mail to the buyers.  The concept has since been copied by several
other bookstores.

\myimage{gr/amazon-1}{The frontpage of Amazon.co.uk.  An ISBN-number
  has been entered in the very prominent search box}{amazon-1}

The Amazon frontpage is shown in figure~\vref{fig:amazon-1}.  Note the
high amount of potentially interesting links, and the prominently
placed search box.  Due to the limited domain of books, the search
engine can make assumptions about what the user is requesting
information about.  In this case a ten-digit number was entered which
happened to be an ISBN-number, which in most other search engines
should have been entered as ``ISBN-1-55860-5347'', and the page for
the corresponding book was returned (see figure~\vref{fig:amazon-2}).

\myimage{gr/amazon-2}{The result of the search in figure~\vref{fig:amazon-1}}{amazon-2}

This page shows the essential information like author, price, and the
number of pages, but also an image of the cover, the rank on the
Amazon sales list, and a direct link to purchase the book (Amazon
remembers your credit card information, allowing for their patented
``One-click-purchase'' technology).  What makes the Amazon web page
truly better that a printed catalogue is that the users are encouraged
to submit feedback, and the information Amazon collects from online
sales to provide information to potential customers about other books
that might interest them, because they interested other customers
which bought the book they are looking at now.

These features are distinct for a page for a purchasable item at
Amazon:

\begin{itemize}
\item \textbf{Average customer rating} - this is the essence of all
  the reviewers opinions, on a scale from 0 to 5.
\item \textbf{Reviews} - readers take the time to create comprehensive
  reviews of the books, all of which Amazon then places prominently on
  the page, with due credit to the individual author.   Recently a
  ``Was this review helpful to you?'' button was added to each review,
  and the result of each poll is shown along with the review, allowing
  a customer to rapidly determine how to rate the review.

\item \textbf{Similar books} - if a previous buyer bought other books
  along with this one, they might be interesting to this person too.
  Links are presented to those books, as well as to the general
  categories this book was placed in by the Amazon librarian.

\item \textbf{Expected content} - Amazon can be expected to have
  \textit{any} English book in their database.  Users expect to be
  able to find a given book in the Amazon database, and usually do.
  Personally I have never looked in vain at Amazon for English books.
\end{itemize}

Amazon explicitely invites those who read the page to review the book
if they have read it, with special treatment for the author and the
publisher.  \textsf{As Greenspun observes, then this is the most
  efficient material}.  At the time of writing, this book has 9
reviews at \textsf{amazon.co.uk}, but 198 at the mother site at
\textsf{amazon.com}.  The corresponding HTML-pages are 22 kb and 550
kb respectively, which gives an an impressive \textsf{Greenspun
  factor} (setting the UK-version to be 100\% \textsf{??-provider}) of
25.

Amazon uses very aggressive marketing, which has caused it to be a
well-known brand name in just a few years.  By giving money to those
who link to their pages, as well as pay search engines to have links
to the Amazon site show up at the top in just about every search, they
have managed to be just about the only Internet bookstore known to
most people.  

Conclusion: The Amazon family of sites are very good, and do what they
can to improve with the help of visitors.  Amazon have been able to
keep their leader status in the virtual bookshop niche, by living up
to the customers expectations.  Hopefully, they will make money one
day.



\section{Bang and Olufsen}
\label{sec:bang-and-olufsen}

\myimage{gr/b-and-o-1}{The Bang and Olufsen pages -- notice the three
  randomly selected images at the top}{b-and-o-1} The
\myurl{http://www.bang-olufsen.com/}{Bang and Olufsen} web-site was
previously very unusual in the way that they offered random
navigation.  Their front page contained very few links, and some
product images which could be clicked and lead to a large set of pages
(see figure~\vref{fig:b-and-o-1}, with three random clickable images
at the top.  These were chosen randomly from the large set of such
pages (600 or more) and put at the top.  This has been highly
efficient in keeping people wandering around and reading their pages,
giving these people an unusual experience.

The underlying database is both used for storing the many pages, but
also for tracking the users as they wander around the site.


\section{Deja - where did Usenet go?}
\label{sec:deja}

\textsf{Write about this if there is time for it}

\myurl{http://www.deja.com/usenet}{Deja Usenet archive}


\section{www.krak.dk}
\label{sec:www.krak.dk}

Making maps on demand was one of the earliest applications for the
web.  The \textsf{web map application was started in 199?} was
discussed in section~\vref{sec:map-generation-first-web-application},
and has since been followed by \textsf{yahoo map, whoelse.}

In Denmark, the well renowned Danish map manufacturer
\myurl{http://www.krak.dk}{Krak A/S} opened their Internet phone
directory and web map generation service in direct competition with
numerous other directory services.  Their advantage was up-to-date
data from the databases of Tele Danmark combined with an ability to
generate a map for a given address, whether it was entered directly or
being a byproduct of a search for something else.  This has proven to
be so powerful a combination of services that all other phone
directory services has been left far behind.

The website has been steadily improved with new
facilities and cross-references, as well as more precise data (In
\textsf{??} 1999 it could not locate ``Herlev Hovedgade 205'' on the
map, so the whole of the road was inked on the map.  This is corrected
today).  The start page is shown in
figure~\vref{fig:krak-1}, and is seperated in two parts; namely white
pages (phone number based), and pink pages (company based).

\myimage{gr/krak-1}{Welcome page (and search form) for www.krak.dk}{krak-1}

The fields in the white page section are Name, Road, House number, Zip
code, City and Phone number.  The fields in the pink page section are
Company and Phone number.  The form has been filled out with a search request
for ``Odense Universitet'', and the results are shown in
figure~\vref{fig:krak-2}. 

\myimage{gr/krak-2}{Result of searching for ``Odense University''}{krak-2}

Each line containing an answer may have icons referencing
to facilities regarding the location of the answer.  These answers
have pointers to 
Rejseplanen, the Route Planner, and a map reference.  Additionally
pointers to a web page, and an email address could have been provided
by the users.  Clicking the map icon of the first line actually
referring to Odense University brings us to figure~\vref{fig:krak-3}.

\myimage{gr/krak-3}{Following the map icon for the first ``Odense
  University'' reference}{krak-3}

The blue dot usually indicates the location of the address but in this
case the map is not 100\% correct (Moseskovvejen does not go all the
way south-east to the parking lot, and the University is located 500
meters further to the south).  When a map is display, the navigation
area below the image allows for zooming and moving the contents of the
shown area.  By selecting ``Zoom out'' and ``x8'' and clicking on the
blue dot a new map is shown (see figure~\vref{fig:krak-4}), where it
is evident that the database show a great deal of detail.  Roads,
residential areas, streams, train stations, the motor way and green
areas are shown.  These maps provide a level of information
corresponding to Krak's printed maps.

\myimage{gr/krak-4}{The map in figure~\vref{fig:krak-3} zoomed out
  with a factor 8}{krak-4}

Now go back to the list of results from the search for ``Odense
University''.  The Car-icon gives a route to the listed location,
where you must fill in the source yourself.  Figure~\vref{fig:krak-5a}
show the form filled in with my own address and the address of the
university, and ``Route on map'' (Rute p� kort) gives a visual route
between the two locations, along with an estimate of the distance and
time it will take.

\myimage{gr/krak-5a}{The entry box where the starting and ending point
  is entered}{krak-5a}

This map is shown in figure~\vref{fig:krak-5b}, and is perfectly
adequate for a person driving a car.  For cyclists it is often a good
idea to look for short-cuts, if you are well known in the area.

\myimage{gr/krak-5b}{The route from my home to the University}{krak-5b}

%\myimage{gr/krak-6}{???????????????????????
%  University'' reference}{krak-6}

If you need step-by-step directions, Krak can provide that too.
Select the ``\textsf{???}'' button and print out the directions.
Careful studies of these directions can give an idea of the network
grid that the route planner use for finding the shortest road.

%\textsf{what is~\vref{fig:krak-6}?}


\section{The Journey Planner - www.rejseplanen.dk}
\label{sec:www.rejseplanen.dk}

DSB (Danish Rail) have a journey planner which is based on train
stations, and major bus stops.  Figure~\vref{fig:dsb-1} shows the initial
form where the two end points for the journey as well as the arrival
or departure time is indicated.  Please note that the two surrounding
frames (left side and top - the scroll bare on the right indicates the
actual area available) leave only about 70\% of the area to the
application itself.

\myimage{gr/dsb-1}{Rejseplanen1}{dsb-1}

\myimage{gr/dsb-3}{Rejseplanen3}{dsb-3}
\myimage{gr/dsb-4}{Rejseplanen4}{dsb-4}
\myimage{gr/dsb-5}{Rejseplanen5}{dsb-5}

The search returns a number of potential journeys shown in
figure~\vref{fig:dsb-3}, with departure and arrival times, total
expected travel time, and the number of changes necessary during the
journey.  The departure corresponding the best for the indicated
period is highlighted with the blue bar.  By accepting the default
selection in the blue bar, and scrolling down to press a button,
figure~\vref{fig:dsb-4} is shown.  This is a direct connection, and
it is possible to reserve a seat directly from this screen.

A more complicated request is shown in figure~\vref{fig:dsb-5} where a
connection from Malmparken (suburb to Copenhagen) to Esbjerg (Western
Jutland) is requested.  A route involving three trains and a bus is
suggested, which is very reasonable and probably the fastest too.

Rejseplanen is helped by the fact that the number of nodes in the grid
can be small compared to \texttt{www.krak.dk}, but the presentation is
not good enough.  It is generally too hard to get to the information
if you need a slightly alternate view from what the programmers
expected.

I used to live next to Odense Sygehus, which is a stop on the
Odense-Svendborg railroad, but where not all trains stop.  The
question regarding whether it would be faster to walk to the Odense
Sygehus station or take my bicycle to Odense Station was very hard to
answer by Rejseplanen, since it does not just simply allow access to
the underlying database.  I ended up calling the station and asking
them to look in their paper copy.




\section{Freshmeat.net}
\label{sec:freshmeat.net}

Freshmeat.net is a service for locating software.  \textsf{a lot of
  products.  users can announce software, and point people toward
  homepages and download pages.  Very nice}


\framepage{15cm}{
\textsf{Check if there still is companies which boost your presence on
  web search engines}

\textsf{Businesses - notably in porno - discovered this and polluted
  their web pages with keywords to boost their chance of being
  returned as a result of a search.}

\textsf{Look for statistics on search engines and ``sex'' - talk about
  that these businesses have a great interest in attracting you since
  they cannot rely on you seeing them elsewhere (like usenet spam)}

\textsf{look for email about somebody who tried google and was pleased
  after being driven away from search engines.   Why does google
  tick?}
}

\myurl{\textsf{http://ftpsearch.lycos.com}}{\texttt{ftpsearch.lycos.com}}
is a unique service provided as a byproduct of the main search engine
at \myurl{http://www.lycos.dk}{www.lycos.dk} where you may search for
\textit{images} on the web.  The search engine does not look inside
the images it finds while traversing the net, but looks at the
filename and the context in the web page for hits.  It works
reasonably well - most of the cactii images for my project was found
this way.  \textsf{Check on details how they do it.}


\framepage{15cm}{
\label{sec:google}
  Search engines.  How do they do it?  www.909.dk, www.krak.dk,
  terraserver.microsoft.com.  Encyclopedia Brittanica.

  ASP+Access, tinderbox \& bugzilla, javasoft (developers area),
  LXR+Bonsai.  ``Alle danskere paa nettet'' - Dansk Journalistforbund,
  Pressemeddelelse. Greenspun selv?
  }

\section{The Collection of Computer Science Bibliographies}
\myurl{http://liinwww.ira.uka.de/bibliography/index.html}{http://liinwww.ira.uka.de/bibliography/index.html}

\textsf{screens hot}

This is a great web-site for scientists because in addition to the
basic search facilities it allows you to download the BibTeX entry for
the books and articles you find.

\section{Jydske Bank}
\label{sec:jydske-bank}

THe only java-solution for home banking.  Any comments?


\section{3coms elektroniske papir}

\textsf{Beskriv projekt fra Jyllandspostudklip}


\section{Yggdrasil - a simple navigational framework}
\label{sec:yggdrasil}

\myimage{gr/yggdrasil-1}{Typical Yggdrasil navigational
  screen}{yggdrasil-1}  

The Yggdrasil system was designed to provide a simple, consistent
navigational framework around a dynamic set of web pages provided by
users, as well as providing a "recently changed pages"-list plus an
overview of pages written by each person.  This was a very successful
idea in the start, but gradually lost momentum due to these factors:

The development platform was changed from Unix to NT, which made it
much more difficult for the users to access their web-directories, as
these were no more a part of their home directory\footnote{The Apache
  webserver uses the ~/public\_html directory for the users personal
  web pages.  The transition to NT meant that the users - in addition
  to their normal file manipulations - should telnet to a Unix server,
  and change the file attributes every time the file was updated.  }

The internal document format became Microsoft Word, which at that time
could not be converted to HTML.  Such documents were therefore unable
to be published.

The users could trigger an update by sending an empty email to the
system, as well as rely on an automatic nightly update.  The trigger
mechanism was not brought along when the email system was converted to
NT.

New users was not informed about the system.



\section{The TOM browser - converting documents online}
\label{sec:tom}

\myurl{http://wheel.compose.cs.cmu.edu:8001/cgi-bin/browse}{The TOM
  browser} is a document conversion system which allows users to
upload files and provide URL's to document on the web, which is then
converted to one of numerous formats.  The default is to autodetect
the file type, and convert to a viewable format.





% \section{Cactus -- document capture and conversion}

% \framepage{15cm}{
% Two out of four methods are implemented (email, printer).  (Increase
% blob limit in DBI::MySQL).  Sample PDF viewer, perhaps simple XLS viewer.

% Demonstrate that Cocoon can provide the navigational framework.

% A full description is available in~\section{site:cactus}
% }
%%% Local Variables: 
%%% mode: latex
%%% TeX-master: "rapport"
%%% End: 

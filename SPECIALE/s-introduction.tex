
\chapter{Overview}

\textsf{This section is rudimentary -- it must be written at the end.
  Currently there is just keywords.}

\textsf{In this thesis I refer to the author Annie, and the layouter
  Larry.}

\section{What is a ``database backed webserver''?}

A web-server is a program which hands out files to any web-browser
that asks.

A database ``is a spread-sheet that several people can update
simultaneously'' (\textsf{greenspun}).

A database backed webserver then is simply a web-server which knows how
to talk to a database server while servicing bypassing web-browsers.

% In order to be useful it must be \textit{fast}!

% A \textit{web server} is a program which waits for requests from web
% browsers (like Netscape Communicator) and returns do

% A \textit{databased backed webserver} is a webserver which has access
% to a database at the same time as it serves documents to clients.
% Such access opens new possibilities, some of which are:

% \begin{itemize}
% \item The database can protect against race conditions when data is
%   updated on the server.
% \item Provide a searching facility more complex than grep.
% \item Allow fast access to data without worrying about limitations in
%   file systems
% \item ``\textsf{users can extract data on their own if they want to}''
% \item Databases are optimized for disk I/O, webservers for network
%   I/O.  Those go well together.
% \item Very large amount of data can be online
% \item Data can be managed remotely by others than the webmaster
% \item \textsf{what else?}
% \end{itemize}


\section{Why use a webserver in combination with a database?}
\label{sec:why-use-a-webserver-in-combination-with-a-database}

In one word: \textit{Synergy}!

The web browser provide a simple, efficient and
well known user interface available to almost every computer user in
the world, and the database provide efficient dynamical access to
data.

The combination of the two allow for the personalized Internet where a
given webserver can give any user customized information upon request.
Examples are:

\begin{itemize}
\item Newspapers - the user may specify that she wants in-depth
  coverage of international affairs, and do not want sports, and her
  personal web-edition will then just contain that.  Payment may be
  per-article instead of per-paper.
\item Maps/Airlines/Hotels - the user may request the best way to
travel from A to B at a given time, see the route, and order
accompanying reservations on-line.
\item Real stores - the user may browse and order from the inventory.
  The ordered goods are then delivered by mail or similar.
\item Virtual stores - the user may browse and order from a virtual
  inventory.  The order is then dissected and forwarded to the
  appropriate suppliers to the virtual store.

\item Banks - there is a growing need from customers to be able to do
  home banking.  If the bank offer a browser based solution, they can
  support customers regardless of their choice of computer.  The
  transactions done through such a system will most likely be stored
  in a database to avoid dataloss.
\item Program development - The \myurl{http://www.mozilla.org}{Mozilla
    project} has shown the benefit of up-to-date information regarding
  just about anything related to the source code.  \textsf{write about
    it}
\end{itemize}


Since its inception in 1993 the web has grown to be the ``good enough
for most things'' graphical user interface, capable of showing text
and graphics as well as provide navigation.  Due to the simplicity of
HTML it is not hard to build a plain basic client\footnote{Plan basic
  client implies no Java, no Javascript, no cookies, no plugins, no
  cascading style sheets, no HTTP-1.1 and almost everything else that
  characterize the modern browsers from Netscape, Microsoft, Suns and
  Opera\textsf{??}.  However, if a webpage is well-written it can still
  be displayed on such a plain basic client. } meaning that just about
any modern computer with Internet capabilites have browsers available
for them.



% \subsection{Protection against race conditions}
% \label{sec:protection-against-race-conditions}

% A database can provide atomic operations on its data, as opposed to a
% stock Unix file system, where data unwittingly can be corrupted if two
% processes try to read and update information at the same time.  The
% concept is well known from multi-programming \textsf{the eskimo book?}

% Companies like ``valueclick'' and ``\textsf{doubleclick?}'' sell
% banner ad's (the \textsf{480x120?} pixel advertisements allowing you
% to click through to the advertisors website if you are interested).
% These are either sold by the number of \textit{impressions} (shown to a
% user) or \textit{clicks} (where the user actually clicks on the
% banner), and these must be counted in order to document that the
% customer gets what he pays for, and the banner ad provider does not
% expose more than what the customer paid for.

% Here is a database system essential in order to protect the data from
% being accidentially overwritten.  ValueClick states that they have
% \textsf{7 million impressions}  daily, which is around 80
% impressions pr second.

% \textsf{And what else is new?}

% \subsection{Provide searching facilities}
% \label{sec:providing-searching-facilities}

% A flat file basically only allow you to search the content and the
% filename with modifications based on the timestamps of the file.
% Searchings normally happen linearly through files and directories.

% Since a database may have a lot more attributes, queries can be much
% more sophisticated, and may even be done against indexed data.

% \subsection{Allow fast access to data without using file systems}
% \label{sec:allow-fast-access-to-data}



% \subsection{????}



% \subsection{Databases are optimized for disk I/O, webservers for network}
%   I/O.  Those go well together.
% \subsection{Very large amount of data can be online}
% \subsection{Data can be managed remotely by others than the webmaster}
% \subsection{Others?}


\framepage{15cm}{
Provide background information: Quick overview of the technology
[webserver, database, http, sql, dynamically generated content vs]
with a graph.  Give simple example.  Explain \textsl{when} things
happen (on-the-fly vs statically generated pages).  Explain the
advantages and disadvantages databases have over flat filesystems
[transactions(protection, atomicity), extra attributes, speed, indexed
columns, complex queries, scalability(linear search in filesystems,
multihost db's),]
}

\section{The new role of the webserver}

\framepage{15cm}{
Compare original functionality with static HTML-files with content and
a few CGI-scripts, to the current dynamically generated sites with
many hits pr day.  Since HTML is generated and provides little
abstraction it is hard to use on a higher abstraction level, and with
unreadable code in ASP and PHP3 it is harder.  CSS was not the answer
since it did not provide any abstraction from the presentation
(seperate content from layout).

Different needs of the user - wap/pdf/html/braille.  CPU, storage,
webmaster time prevents all formats being pregenerated.
}

Documents must be written in a format describing \textit{content} and
not layout.


%%% Local Variables:
%%% mode: latex
%%% TeX-master: "rapport"
%%% End:

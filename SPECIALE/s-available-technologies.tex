
\chapter{Overview of technology available for Linux February 2000}
\framepage{15cm}{
Describe what products there are currently available for these
categories. Evaluate as much as possible.   Conclude that currently
this is an area in great development.
}

\section{Webservers}
\framepage{15cm}{
Apache, misc Servlet Javaservers (jetty, thttpd, java webserver,
etc).  Look for comparisons.  Roxen (graphs).  Discuss whether several
webservers should be run on the same machine to provide better services.
}

\section{Database engines}

% \framepage{15cm}{
% Full SQL servers:  MySQL/mSql/DB2/Oracle (spank their FLUG
% policy)/Informix/PostgresSQL (discuss features:  transactions, unique
% numbers, in-kernel code, speed, availablity, multi-host
% capabilities?). Any flat file RDBS?

% Tricks: Menu system in LDAP (lr.dk).
% }

The number of database vendors which support Linux (which as for now
is \textit{the} Open Source operating system which people know about)
is steadily increasing.  \myurl{http://linas.org/linux/db.html}{This
  webpage try to keep track of them all}, but I only list the major
commercial vendors and Open Source projects, where the drivers should
be of production quality.  The following overview was created March
2000.

\framepage{15cm}{
  PostgresSQL (discuss features:  transactions, unique
  numbers, in-kernel code, speed, availablity, multi-host
  capabilities?). Any flat file RDBS?
  }

\subsection{Cloudscape - Informix}
\label{sec:cloudscape}

The
\myurl{http://www.informix.com/informix/press/1999/dec99/cloudscape.htm}{Cloudscape
  Database} is written in Java, with a JDBC and a HTML interface, has
many core SQL and Java extensions implemented, and allow distribution
amongst several Java-machines (using RMI).  Informix bought Cloudscape
Inc. in October 1999 to get this database system, so it is still very
new.  They offer
\myurl{http://www.cloudscape.com/Evaluations/index.html}{a 60 day
  trial version from their website}.  Pricing is approximately \$900.


\subsection{DB2 - IBM}
\label{sec:db2}

\myurl{http://www-4.ibm.com/software/data/db2/linux/}{DB2 for Linux}
is available - the \myurl{DB2 Personal Developer's Edition
  V6.1}{http://www6.software.ibm.com/dl/db2pde/db2pde-p} is available
for free for non-commercial purposes.  

\myurl{http://www-4.ibm.com/software/data/db2/extenders/xmlext/}{DB2
  XML Extender} which ``let you store XML documents in DB2 databases
and new functions that assist you in working with these structured
documents.''.  It is available for AIX, Solaris and Windows NT.
  \textsf{Evaluate?} 

  

\subsection{Informix - Informix}
\label{sec:informix}

http://www.informix.com/datablades/dbmodule/informix1.htm

 
\subsection{Ingres - Ingres}
\label{sec:ingres}

The
\myurl{http://www.cai.com/products/betas/ingres\_linux/ingres\_linux.htm}{Ingress
  II database is in a beta stage for Linux}, and can be downloaded for
free.  It uses Perl, gcc and Apache to provide webserver facilities.


\subsection{mSQL - ?}
\label{sec:msql}

\textsf{Yes?  Looook for stuff - rememebr to get the mSQL/MySQL book
  in the databse}

\subsection{MySQL - TCX}
\label{sec:mysql}

\myurl{http://www.tcx.se}{MySQL is an Open Source database} which is
very popular amongst Perl programmers due to
\myurl{http://www.tcx.se/benchmark.html}{the good performance}
combined with the free availability of source code and high quality
drivers.

MySQL only costs money if you run it on a Microsoft operating system,
or commercially on a server.

\subsection{Oracle}
\label{sec:oracle}

The complete Oracle 8i product range is available for Linux, with a
low-end version available for download.  The WebDB tool functions as a
webserver-interface to any Oracle database.

On \myurl{\textsf{??}}{the FLUG meeting ??} we were promised that we
could have all the Oracle software for Linux we needed (without
official licenses), and that Oracle Denmark would provide the
necessary support.  It is interesting that Oracle Denmark unofficially
seed the Linux communities with pirate copies of their software.

\subsection{SyBase - Sybase}

Sybase \myurl{http://www.sybase.com/products/linux/}{has Sybase
  Adaptive Server Enterprise and SQL Anywhere Studio} available for
  Linux for free as long as they are used for development.  For
  production installations a license must be bought.  An earlier
  version is unsupported but may be used without restrictions for
  production enviroments.
  
  Their web integration tool does not appear to have been ported to
  Linux yet.  The \myurl{http://linas.org/linux/db.html}{Linux
    database page} lists several third-party tools which provide web
  functionality.

  Even though Microsoft does not provide Linux drivers for the
  Microsoft SQL-server, the freely available SyBase drivers for Linux
  appear to be usable.

\subsection{Other approaches to database storage}

\myurl{http://www.lr.dk}{Landbrugets r{\aa}dgivningscenter} uses a
LDAP-server\footnote{\textsf{Light Directory Access Protocol?} - me
  thinks} to provide easy access to commonly used web elements.

\textsf{Others?}


\section{Browsers}
\framepage{15cm}{
MSIE5 (XML support, Solaris), Mozilla (alpha), Netscape (too old),
HotJava 3.0 (no XML, can be expanded to do XML conversion
internally?), Opera (?), Amaya (other W3C browsers?), AWTviewer in fop
(can it do xml directly?),
}


\subsection{Netscape Communicator}
\label{sec:netscape-communicator}

The Netscape Communicator browser is available and fully supported for
Linux.  The version available at March 2000 was 4.72 in English only.

This browser have a reputation of being rather picky of its
environment.  Especially the Java-subsystem is prone to crashing the
server.  If Java is important, consider HotJava.


\subsection{Mozilla - Netscape 6.0}
\label{sec:mozilla}

\myurl{http://www.mozilla.org}{The Mozilla browser} has been under
development since 1998, where Netscape -- inspired by the \textsf{the
Cathedral and the Bazaar} paper -- decided to switch development
strategy for its incomplete source for Netscape Communicator 5.0 from
an in-house system to a full-scale OpenSource model housed at
http://www.mozilla.org.

In the past two years, the OpenSource developers have basically
rewritten the Communicator into a truly impressive product (this is
based on the M14 milestone release).  A tentative release is midsummer
2000.


\subsection{Opera}
\label{sec:opera}

This Norwegian browser from \textsf{??} has been highly successful on
Windows for providing a good browser to smaller, older machines.  They
have \textsf{a beta port under development what is the status?  XML?}.  


\subsection{HotJava 3.0}
\label{sec:hotjava}

HotJava is a Netscape 3.0 compatible browser written in Java.

It is a bit slow, especially in updates, but has its force in its
ability to execute Java applets (which is where Netscape is slow, and
Microsoft does not comply with the standards).

HotJava requires a Sun JDK-based Java implementation (clones do not
provide all the functions they need), and I have tested version 1.1.2,
1.1.4, and 3.0 under Linux, OS/2, Windows, Irix and Solaris, where it
worked satisfactory.

\textsf{There is no information regarding the status of parsing XML
  internally.  Full access to the source is probably necessary before
  any third-party implementations come}


Rumours has it that development on this browser has ceased.  I have
tried to extend version 1.1.4 to show TIFF files, but had to give up
due to the miserable documentation combined with lack of the source.

Other rumours has it that Sun will release the source for HotJava
under their Community License (which basically makes the developers
unpaid employees of Sun) like it has been done previously.  On March
13 Sun released their first product under a true OpenSource license
(the Forte product - previously Netbeans), which could indicate that
they had evaluated their experience with previous releases, and found
that nothing short of full OpenSource will do.

Today there are so many OpenSource projects that the truly talented
coders can choose freely what they want to develop on, and they choose
projects where they can get return in recognition from their equals,
and where they don't feel that they slave for a company.

Sun has a ``vote for the bug you want fixed'' system for
Java\textsf{URL}, and the highest rated bug was ``Port Java to Linux''
for a \textit{very} long time, and with four times the number of votes
for number 2.  The Blackdown \textsf{Team} ported Java 1.1 to Linux,
and collaborated with Sun on porting Java 1.2 to Linux.  It took very
long time, and when they finally had it officially published on the
Sun Java webpages, Sun did not give credit to the Blackdown Team - not
on the web pages, not in the code.  The Blackdown Team was highly
offended, said publicly so, and stopped working on the port!

My personal guess is that this incident taught Sun that they must be
very observant to the OpenSource culture they want to attract.  It
must be nurtured and credited in order to create the symbiosis that
the Mozilla project has demonstrated to be possible.

\subsection{Microsoft Internet Explorer 5.0}
\label{sec:microsoft-internet-explorer}

Microsoft has created a very nice browser which works well on Windows
platforms, and which supports a recent version of XML with the
XSL-stylesheets (\textsf{reference)}.  Since the XSLT standard was
released January 2000, a release of MSIE50 can be expected ``rather
soon'' which implements the XSLT standard, allowing the browser to
render XML files on the client side.  For now, server side processing
is still the only option.

Microsoft Internet Explorer 5.0 is \textit{not} available for Linux.
It is, however, available for Solaris and HP-UX so the codebase is
ported to Unix, and from there it is usually comparatively easy to get
software to run under Linux.  The Solaris version was incompatible
with the window manager I was running under Irix, so I have not tested
it much, but it was very, very similar to the Windows version.

My personal guess is that Microsoft holds back Linux versions of the
Internet Explorer with Outlook Express until they see a fortunate
opportunity to release it.  \myurl{\textsf{??}}{A certain lawsuit
  springs to mind} - I do not expect Microsoft want to be accused of
monopolizing the Linux market too.

To summarize:  Microsoft Internet Explorer is not available for Linux
at the time of writing.

\textsf{Try Solaris version of MSIE under KDE}

\subsection{Amaya}
\label{sec:amaya}

Amaya is the W3C testbed for HTML-development.  Is a nice browser
combined with a reasonable editor.  \textsf{Does not do XML at all}.
\textsf{Don't they have a testbed?}

\subsection{AWTviewer in fop}
\label{sec:awtviewer}

\textsf{probably for fop stuff only}


\subsection{StarOffice}
\label{sec:staroffice}

\textsf{What can we do here?} - try i tout.

StarOffice include a Netscape \textsf{3.0} compatible browser, with
Java applet support.  It is an integrated part of the StarOffice
program, which besides Linux is available for Windows, OS/2 and
Solaris.


\subsection{\textsf{OTHER BROWSERS?}}



\section{XML utilities}
\framepage{15cm}{
XT, Xerces, Cocoon, SQL2xml, validators (error recovering?),
converters to XML (pod2docbook, tex4h, Perl with appropriate modules?,
wordview modified)?

Show the XML->SQL->XML->HTML on the fly conversion possible with
Cocoon .... Without auxillary code. All done in W3C style.
}





\section{PDF utilities}
\framepage{15cm}{

PDF generated by pdflatex is good.  PDF generated by fop isn't.
createpdf.adobe.com 

pdfzone.  acroread (ok, heavy), xpdf (font problems with bitmaps, ugly
but fast, gives best result with usepackage{times}, no anti-alias), gv
(pretty good).  Distiller (check what it can do).  Use pdfinfo to get
the number of pages in the pdffile.

Conclude that it is possible to have a 100\% Java solution for
on-the-fly publishing XML to HTML and PDF through a webserver.
Discuss why developers do it in Java, as opposed to any other language.

Very few utilites support creating PDF.  Discuss how to create PDF.

}

%%% Local Variables: 
%%% mode: latex
%%% TeX-master: "rapport"
%%% End: 

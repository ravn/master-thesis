% $Id$

\chapter{?? - process documents now that we have them anyway}

\section{The consequence - multiple views of a document}
\emph{Please read the ``Terms and Concepts'' before continuing.}

\framepage{15cm}{
Introduce publishing where HTML is just a backend amongst many (PDF,
Word, ASCII).

Describe why it is important to be able to render finished versions of
documents fully automatically from a single \textsl{annotated} source
(www/print/cdrom/Palm Pilot/braille/handicapped persons,whatever).  The better
the annotation, the better the output (ref: Stibo).  Describe the need
for SGML (history/usage) and XML (why/browser support/on-the-fly
publishing/bleeding edge - being standardized).

Donald Knuth - {\TeX}/Web/Literate Programming - advantages (high
quality, excellent math, superb algorithms, can be tailored to needs
(basic interpreter written in TeX) and
disadvantages (programming language, not abstract) (designed 20 years ago).  Ask Steffen Enni
about his thoughts.  Javadoc.  Perl POD.  DocBook projects (FreeBSD,
LDP).  
}

\section{The user should use current tools to publish documents}
\framepage{15cm}{
Publishing a document to the web should be as easy as printing a
document.  The basic principles are the same - you can do with a
``Publish''-button and a dialogue box where the essential information
is provided.  It is not so today - discuss reasons .  Use ``print to
fax'' software as horrible example of this idea gone wrong.

When the webserver is dumb, you cannot do much.  If there is a
database underneath, much more is possible.  Discuss the idea of
having several ways of entering documents in the database (upload via
form, send as email (fax software can do this too), print to virtual
printer, scan, fax, voice) and letting the software do the
conversions.

Use example with print PostScript to file and upload via ftp to
printer (LexMark).

Writing XML directly is much harder to do than writing HTML (stricter
syntax, more options, plainly just more to type), and should be aided
by a good tool.  Alternatively, the user should use well-known tools
(Word) and mark up according to very strict rules, which is then
automatically converted to the XML document.  This does not give as
rich documents, but allows users to publish existing documents with
very little trouble.
}


\subsection{The importance of a web cache}
\label{sec:the-importance-of-a-web-cache}

A database query is expensive, and it requires an expert to tune the
database to run as fast as possible.  It is not, however, always
necessary to have the webserver do a database query to serve a page -
often the generated page is valid for a short or long term period,
and then it is relatively easy to cache the page for this period.

Here is one way to do it:

\begin{itemize}
\item Configure the script generating the page, to add an
  ``\texttt{Expires''} header with a reasonable time of expiry
\item Set up squid (\vref{sec:squid}) in http-accellerator mode, where
  it transparently adds cache facilities to a webserver, respecting
  the ``\texttt{Expires}'' header.
\end{itemize}

%%% Local Variables: 
%%% mode: latex
%%% TeX-master: "rapport"
%%% End: 

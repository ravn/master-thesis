% $Id$

\chapter{The DocBook and TEI DTD's}

\section{DocBook considerations}
\label{sec:docbook}

Having discussed the merits of SGML and XML, the next step is finding
and deciding on a suitable DTD for the documents.  My findings are
that there currently is two trends on the Internet regarding
DTD-development:

\begin{itemize}
\item Small DTD's designed for a specific purpose
\item Large DTD's designed for a wider array of possible applications
\end{itemize}

Since we are interested in translating to and from this document
format, and we prefer not to have to maintain these translations
ourselves, it is only the large DTD's that are interesting.

I have been looking for such large, well-maintained DTD's on the
Internet for almost a year now, and this is an extra-ordinarily rare
thing.  I have basically only found three possible candidates:

\begin{description}
\item[\myurl{http://www.docbook.org}{DocBook}] -- A DTD designed for
writing technical documentation.  It is being used by a steadily
increasing number of companies and OpenSource projects.


\item[\myurl{http://www.uic.edu/orgs/tei/}{TEI}] -- Text Encoding
  Initiative.  A DTD intended to be used with ``preparation and interchange of electronic texts for scholarly research, and to satisfy a broad range of
uses by the language industries more generally.''

\item[\myurl{http://www.openebook.org/faq.htm}{eBook}] -- the eBook
  format is intended for ``Authors, editors, publishers and content
  owners who want to have their titles in a format that is "eBook
  ready," which then can be used by a variety of electronic book
  publishing systems and reading devices.''
\end{description}

% \textsf{
% The SGML is converted to HTML by applying a HTML style sheet for
% DocBook, and sending it through jade with request for a sgml
% conversion.  The stylesheet generates a HTML file pr section in a
% chapter which is rather too much.  The HTML is \textit{very} ugly - all
% induced line breaks are \textit{within} the tags to avoid introducing any
% artificial whitespace.  The HTML is Lynx-compatible.


I have concentrated on the DocBook format simply because it had the
best documentation (DocBook: The Definitive Guide which was published
in late 1999 \cite{walsh-muellner:docbook-the-definitive-guide}), and
appeared to fulfill the needs for the document conversion utilities I
wanted to have.

Even though DocBook originally was designed for SGML, the principal
developer Normal Walsh, has done a lot of work on developing two XML
versions -- one with full functionality, and the other with a limited
subset intended for rendering inside a XML-capable browser.   This is
intended to be fully present when the official version 4.0 version of
DocBook is released around third quarter 2000.




% nice in Netscape.  A ``up'', ``next'' and ``previous'' are available
% combined with a ``Top title''-$>$''Chapter title'' -$>$ ``Section title''
% navigation bar at the top.

% The localization code is manually maintained.

It is possible to have a DocBook XML-file which can be processed both
with XSL-stylesheets \textit{and} DSSSL-stylesheets, by letting the
!DOCTYPE header point to the DocBook XML DTD along with
system-dependent path (which is required by XML).  A sample header for
DocBook is listed below:

\begin{verbatim}
<?xml version="1.0" encoding="ISO-8859-1" standalone="no"?>
<!DOCTYPE article 
  PUBLIC "-//Norman Walsh//DTD Simplified DocBk XML V3.1.3.6//EN" 
  "/home/ravn/sgml/docbk/db315/docbookx.dtd">
\end{verbatim}

If the \texttt{standalone="yes"} field is set in the \tag{?xml?}-tag
instead, it means that the XML-tools will not validate the contents
against the DTD, allowing both for faster performance, but also
without requiring anything installed on the rendering machine but the
style sheets.  Since a full installation of DocBook with all
supporting DTD's and entity-lists is a large and complex affair, this
is a very pleasant feature.

\begin{table}[htbp]
  \begin{center}
    \begin{tabular}[tb]{|l|p{12cm}|}
\hline\hline
Format & Command \\
\hline
  XHTML-1.0 & java com.jclark.xsl.sax.Driver \$$<$ ../docbook/xhtml/docbook.xsl \$@\\
  HTML-4.0 & java com.jclark.xsl.sax.Driver \$$<$
  ../docbook/html/docbook.xsl \$@ \\

  XSL-FO & java com.jclark.xsl.sax.Driver \$$<$ ../docbook/fo/docbook.xsl \$@ \\
  RTF & jade -t rtf -d .../docbook/print/docbook.dsl .../jade/pubtext/xml.dcl \$$<$\\
  MIF (Frame) & \\
  {\TeX} & Use \texttt{jade} with the \texttt{-t tex} option.  Note
  that the special {\TeX}-dialect \texttt{jadetex} must be used to
  process this output.\\
  
  PDF (from FO) & \texttt{tex '\&pdffotex' document.fo}  (Note:  This
  makes much prettier output than FOP)\\
  
  PDF (from FO) & \texttt{java org.apache.fop.apps.CommandLine \$$<$
  \$@}   \\
  
  OThers? & \\
\hline
    \end{tabular}
    \caption{Output formats possible for a DocBook XML document}
    \label{tab:output-formats-possible-for-a-docbook-xml-document}
  \end{center}
\end{table}

Table \myvref{tab:output-formats-possible-for-a-docbook-xml-document}
lists several of the possible formats which a DocBook document can be
rendered to.  See \myvref{sec:xml-publishing} for a functioning set of
Cactus filters implementing a DocBook publishing engine.  

Most of these transformations are currently rather slow, currently
making them unsuitable for ``on-the-fly'' rendering.  XSL-conversions
are feasible (those implemented in Java), but the
DSSSL-transformations are simply too slow.


% \section{TEI -- \myurl{http://www.uic.edu/orgs/tei/}{Text Encoding
%     Initiative}} --
% \label{sec:tei}



% \section{Converting documents to other formats}
% \label{sec:docbook-converting-documents-to-other-formats}


% \subsection{SGML to XML}
% \label{docbook-converting-sgml-to-xml}

% \begin{alltt}
%   sx -biso-8859-1 -xlower input.sgml $>$ output.xml
% \end{alltt}

% The DTD for the document \textit{must} be registered in the
% \textsf{catalog}.  Even so a number of warnings and errors may be
% printed, since some SGML-constructions cannot be represented in XML,
% and some entities are unknown.  This must be handled manually - SX is
% not a tool do this perfectly.


% \subsection{XML to RTF}
% \label{sec:docbook-xml-to-rtf}

% Use \texttt{jade} with the appropriate stylesheet and \textsf{....}

% \begin{alltt}
%          jade -t rtf -d docbook.dsl xml.dcl inputfile.xml
% \end{alltt}

% where \texttt{docbook.dsl} is one of the stylesheets from the
% \textsf{DocBook Modular StyleSheets}
% (\texttt{dsssl/docbook/print/docbook.dsl}) and \texttt{xml.dcl} comes
% from the Jade distribution (\texttt{pubtext/xml.dcl})

% The resulting RTF-file is compatible with Word-97, Word-95 (claimed by
% author), StarOffice, \textsf{other texteditors} 

% \textsf{other backends}

% SQL to XML :: Coocncococns sql processsor

\section{A sample DocBook XML document}
\label{sec:amanda-readme.xml}

I have marked up a DocBook XML version of the
\myurl{ftp://ftp.amanda.org/pub/amanda/README}{README file for the
  AMANDA backup system}, as part of the general transition to an
SGML-based documentation set for AMANDA.  I have used it as a test
document for the XML-publishing sample filter-set in
\myvref{sec:xml-publishing}.

There are a few caveats for HTML-knowledgable people:

\begin{itemize}
\item Each paragraph must be enclosed in a \tag{para}-tag.  Also
  inside a \tag{listitem}-tag.
\item \tag{sect1}, \tag{sect2} etc. encloses around the \textit{whole} section, and the
  \tag{title}-tag shows the title of the section.
\end{itemize}  
In order to be processable by SGML tools like Jade (to get RTF output)
it is necessary to have a \tag{!DOCTYPE} entry, which again requires
-- for XML compatibility -- that the DTD is explicitly given.  This
particular document conforms to
\myurl{http://www.nwalsh.com/docbook/simple/index.html}{the
  "Simplified" DocBk XML DTD}.



\listinginput{1}{../sample/AMANDA-README.xml}

%%% Local Variables: 
%%% mode: latex
%%% TeX-master: "rapport"
%%% End: 

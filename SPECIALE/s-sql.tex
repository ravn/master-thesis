% $Id$

\chapter{Standard Query Language, Databases and Webservers}

\mycitation{You got to be carefully taught}{South Pacific}{Rogers and
  Hammerstein} 

\section{SQL}

SQL (the Standard Query Language) was developed in the early 1970'es
by Don Chamberlin and Ray Boyce at IBM Research, to implement the
mathematical notation in which E. F. Codd had defined relational
databases~\cite{a-relational-model-of-data-for-large-shared-databanks}.
Today SQL has grown into the \textit{de-facto} standard for
communicating with a relational database, which is being enforced with
the modern ODBC and JDBC drivers which provides a ``tunnel'' between
the client program and the database where the client can send a
SQL-command and get the result back, without caring about \textit{how}
the data are transmitted.

Unfortunately this is as much abstraction as you will get from these
``SQL drivers'' -- the ability to execute a SQL-command and retrieve
the result.  Any further abstraction from plain SQL is still left to
the application programmer which means that even the state-of-the-art
applications talk to the database in a command-line fashion.   The SQL
looks like this (liftet from the MySQL tutorial):

\begin{verbatim}


CREATE TABLE shop (
 article INT(4) UNSIGNED ZEROFILL DEFAULT '0000' NOT NULL,
 dealer  CHAR(20)                 DEFAULT ''     NOT NULL,
 price   DOUBLE(16,2)             DEFAULT '0.00' NOT NULL,
 PRIMARY KEY(article, dealer));

INSERT INTO shop VALUES
(1,'A',3.45),(1,'B',3.99),(2,'A',10.99),(3,'B',1.45),(3,'C',1.69),
(3,'D',1.25),(4,'D',19.95);

SELECT * FROM shop;

\end{verbatim}

\begin{figure}[htbp]
  \begin{center}
    \begin{tabular}{|r|l|r|}
\hline\hline
 article & dealer & price\\
\hline
    0001 & A      &  3.45\\
    0001 & B      &  3.99\\
    0002 & A      & 10.99\\
    0003 & B      &  1.45\\
    0003 & C      &  1.69\\
    0003 & D      &  1.25\\
    0004 & D      & 19.95\\
\hline
    \end{tabular}

    \caption{The output from the SQL-example}
    \label{fig:mysql-tutorial-example-output}
  \end{center}
\end{figure}
which then returns the table shown in figure~\vref{fig:mysql-tutorial-example-output}:

\begin{verbatim}
\end{verbatim}

The table shown here was originally in ASCII; inside an application
program it is common to take a row at a time retrieving a structure
with the content of each field.


\subsection{SQL - really a standard?}
\label{sec:how-standardized-is-sql}

SQL has evolved over the years.  The latest ANSI standard is SQL92,
which is discussed in~\cite{understanding-the-new-sql}, \textsf{as
  well as the standard bodies online}, and it provides several levels
of compliancy a given SQL-database can adhere to.  Even so, there are
several facilites which is not covered by the standard:

\begin{itemize}
\item \textbf{unique numbers} -- having a row or function which
  generates a unique number every time it is called.  This is called
  \textit{sequence numbers} in Oracle, and \textsf{?} in MySQL.
\item \textbf{date and time functions} -- querying for the current
  date and handling date/time fields in the database is not standard.
  Additionally it is common for a vendor to have additional formats
  for timestamps which is incompatible with other vendors.
\item \textsf{others?}
\end{itemize}

These vendor specific extensions from SQL92 are normally done in order
to allow faster database queries, and are therefore interesting for
the application programmer who wants the fastest possible
implementation.  Unfortunately this usually ties the application
dependent to a specific database server, which rarely is in the
interest of customers (who might have a database already they wish to
use), meaning that the application programmer must be aware which
facilities are acceptable to use.  I have not found application
frameworks which provide a generic SQL interface and parse it to the
underlying database.



\subsection{SQL - getting started}
\label{sec:sql-getting-started}

I was reintroduced to SQL rather abruptly in October 1999, where I
needed one or more good introductory texts for ASP-programming with
MS-Access as the database backend\footnote{I learned quite a lot in
  those few days.  Basically ASP is a very good idea, and Frontpage is
  extremely unsuited for working with it}, and found almost none on
the Internet.  Martin Damsbo, who had previously written the
PHP3+MySQL based WACO system~\cite{mez-waco}, pointed me towards
\myurl{http://sunsite.auc.dk/mysql/Manual\_chapter/manual\_Tutorial.html\#Tutorial}{the
  tutorial in the MySQL reference manual}, which was what he had used.

The MySQL tutorial is reasonable for a person knowledgeable in SQL to
get acquainted with the MySQL database, but it falls a bit short of
being a generic introduction.

I decided to dvelwe into MySQL since it had excellent Perl support,
which I had chosen for developing the Cactus system
(see~\vref{sec:cactus-choice-of-language} for reasons), and was
OpenSource software available on Linux.


A \textit{much} better choice for me was
\myurl{http://photo.net/sql}{Phil Greenspuns SQL-introduction}
(see~\cite{greenspun-sql}) which was specifically written as
introductory material for college students, based on his long-term
real-life experience with the heavy duty web-service at
\myurl{http://www.arsdigita.com}{ArsDigita}, which also serves
\myurl{http://photo.net}{photo.net} where this SQL-introduction
resides..

Greenspun start with the basics and moves to advanced
topics in a steady pace, and is an excellent read.  There is a strong
emphasis on Oracle 8i, but is easily adapted to other databases like
MySQL.  He has numerous examples, and several references and
evaluations of other texts.  Additionally the text is sprinkled with
thumbnail photos, which links to a large version along with a link to
another part of his site.

This in addition with the comments from the readers to the text, makes
this a very valuable resource for users.  The SQL-introduction is
available both online and as a book.


\section{Some of the possible ways of accessing a database from a webserver}
\label{sec:accessing-a-database-from-a-webserver}

Armed with the necessary, solid SQL knowledge, it is reasonable to
consider the current approaches of interfacing a Web server with a
SQL-database.  At least the following options exist:

\begin{description}
\item[ODBC] -- The Microsoft ``\myurl{http://www.microsoft.com/odbc}{Open Database Connectivity}'' which
  provided for the first database access standard on Windows.  Since
  superceeded by OLE DB which does basically the same thing on Windows.
\item[JDBC] -- The ``\myurl{http://www.javasoft.com/jdbc}{JDBC Data
    Access API}'' does the same thing for Java-programs as ODBC do for
  Windows-programs.
\item[Perl DBI] -- The ``Perl Database Interface'' allows Perl
  programs to speak to Adabas, DBMaker, Fulcrum, Informix, Interbase,
  Oracle, Solid, Empress, Sybase, Illustra, Ovrimos, PostgreSQL,
  QBase.
\item[Python DB-API] -- A rapidly growing set of
  \myurl{http://www.python.org/topics/database/modules.html}{database
    interface modules for Python}.  Currently are Informix, Interbase,
  MySQL, Oracle, Sybase and the generic ODBC interface supported.
\item[Vendor dependant] -- Each vendor have their own way of moving
  HTML (and some XML) in the database itself.  These are usually not
  portable, and therefore not interesting from a generalist point of
  view.  The databases surveyed 
  
\end{description}


\section{What databases are available?}
\label{sec:what-databases-are-available}



%%% Local Variables: 
%%% mode: latex
%%% TeX-master: "rapport"
%%% End: 

% $Id$


\chapter{Dynamically generated documents}
\label{cha:dynamically-generated-documents}

% \framepage{10cm}{Abbreviations are explained in ``Terms and
%   Concepts'' (chapter \vref{cha:terms-and-concepts}).  Please refer to
%   it to clarify matters.}

\mycitation{\textsf{404 Document not found}}{\textsf{?}}{\textsf{?}}

!!! PUT SAMPLE OF FILENAME INSTEAD OF TT-FILENAME-/TT in SOMEWHERE.


\section{The way it started: Handmade web-pages in a filesystem on a web-server}

Back in the old days when the world wide web was designed, a web
server could basically do only two things:



\begin{itemize}
\item Serve \textit{static} files directly from an underlying file
  system.  These files were either pages written in HTML, or images in
  GIF or JPEG formats.
  
\item Call a program complying to the \textit{Common Gateway
    Interface} with user-submitted parameters, and return its output
  as the document to send.
\end{itemize}

That was all!

\begin{my-detour}

Such web servers
  still exist - for example \myurl{http://hoohoo.ncsa.uiuc.edu/}{the
    NCSA server} (which has been unmaintained since 1996. NCSA
  recommend that the Apache server should be used instead).  Even so
  \myurl{http://www.netcraft.com/Survey/Reports/0002/}{the February
    2000 Netcraft survey} reported that 17172 NCSA servers were
  active, and could be reached from Netcraft.
  
  In August 1995 the number of
  \myurl{http://www.netcraft.com/Survey/Reports/9508/ALL/}{webservers
    running the NSCA server was 10835}, meaning that number has
  increased slightly in that period.  For comparison the total number
  of public webservers on the Internet in the same period grew from 19
  thousand to 11 million, with Apache and Microsoft Internet
  Information Server accounting for 8.8 million of these.   
\end{my-detour}


Even with such a simple model, it works well for even \textit{very}
large web-sites which only generates few documents dynamically.

A well-tuned server which serves documents directly from the
filesystem can reach impressive numbers.
\myurl{http://www.acme.com/software/thttpd/benchmarks.html}{ACME
  software has a webserver benchmark} from 1998, which have
benchmarked several webservers.


On a modern PC Apache has no problem saturating
a 10Mbps Ethernet connection (\textsf{what about a 100Mbps
connection)}, meaning that even for very large and demanding sites the
bottleneck will be determined by the
hardware!\footnote{\textsf{Greenspun talks about SCHLOW
    webdatabaserservers - write a bit about that}}

On a related note: The ftp-server \texttt{ftp.cdrom.com} serves a lot
of files every day.  On March 14 2000
\myurl{http://www.emsphone.com/stats/cdrom.html}{the transfer
  statistics} said that the average output that day was 79.3 Mbit/s
(with a peak of 95.8 Mbit/s).  The average output on a yearly basis
was 86.2 Mbit/s, which is 933 Gb/day in average).
\myurl{ftp://ftp.cdrom.com/.message}{This single machine runs FreeBSD}
(see \myurl{http://www.freebsd.org}{www.freebsd.org} for details).
Since an ftp-session have a notably larger overhead than an
http-session, it is not unreasonable to expect similar performance
from a suitably tuned web-server when serving static files.

CGI-scripts is another matter.  For the occasional dynamically
generated document CGI-scripts turned out to work well.  All kinds of
documents -- images, webpages, progress reports -- could be generated
comparatively easy in the favorit language of the CGI-script author.
Libraries to deal with the decoding of parameters, and encoding of the
generated result were soon abundant, providing for many enhancements
to the original ``static web site'' model.

\myimage{gr/xerox-1}{The original Xerox PARC Map Viewer}{xerox-1}
\label{sec:map-generation-first-web-application} One of the first
demonstrations of dynamically generated images was the
\myurl{http://mapweb.parc.xerox.com/map}{Xerox PARC Map Viewer} (see
figure~\vref{fig:xerox-1}) which allowed the user to navigate a
virtual atlas, where each ``window'' to the map was generated on
demand.  This initial quick hack has since been superceeded by
professional map companies which produce maps of an uneven quality on
demand, like \myurl{http://maps.yahoo.com/py/maps.py}{Yahoo Maps} (USA
only) and \myurl{http://www.mapquest.com}{MapQuest} (covers Europe
too).


% \textsf{a reference to the problems with CGI?  From CGI.pm perhaps}

The problems with CGI-scripts is due to several reasons:

\begin{itemize}
\item \textbf{Script run as a subprocess} -- the script is executed
  with the same permissions as the webserver itself, including access
  to \texttt{/etc/passwd} and other possibly sensitive files.  Such
  scripts had to be \textit{trusted} or - for ISP's - inspected before
  installation to ensure that the script would behave properly.

  This has been addressed with the development of ``safe languages''
  where the execution environment is guaranteed that the CGI-script is
  confined to a ``sandbox''.
  
\item \textbf{Incompatible platforms} -- the CGI-programmer may not
  have access to a C-compiler which can generate binary code for the
  machine running the webserver.
  
  This has caused interpreted languages like Perl to be very popular.
  Perl programs can be run immediately on a given platform without
  needing to be recompiled, and are for most purposes as fast as the
  equivalent programs written in C or C++.  Recently Java Servlets
  (see section~\vref{sec:java-servlets}) have appeared as a popular
  and well-supported alternative to CGI-scripts, which is being
  supported by more and more web servers.
  
\item \textbf{Slow} -- the overhead just for invoking the CGI-program
  in a subprocess is substantial even for moderate load on the web
  server, since the web-server must ``fork'' a new process for each
  request.

  This has been addressed by putting the execution environment inside
  the web server, using threads instead of processes, and caching
  compiled versions of scripts.  

\end{itemize}

Several solutions to the above problems have emerged.  Of these, Java
Servlets are discussed in section~\vref{sec:CGI-servlets}, PHP3
scripts in section~\vref{sec:CGI-php3}, and Perl (with the mod\_perl
acceleration module) in section~\vref{sec:CGI-modperl}).

Server side scripting never really took off before the PHP3 and ASP
languages provided ``persistent connections'' to a database servers,
which was unavailable to the standard CGI-scripts.

Persistent connections provide a real performance boost!  Database
connections are notoriously ``expensive'' to establish, so needing to
open one for each and every CGI-script were a true performance
bottleneck.  Just by keeping the connection open (and for ASP-scripts
- retain the particular database connection for the session) was
extremely beneficial.  That combined with a very simple way to switch
between code and HTML plus that every user could use it without
needing to ask the webmaster to install a potentially dangerous
CGI-script, meant that the ability to generate webpages dynamically
became available to everyone.



\section{When a site grows, it becomes hard to maintain}
\label{sec:when-a-site-grows-it-becomes-hard-to-maintain}

\framepage{15cm}{This section needs to be written.
  
  Talk about the growth of the net, and broken links, both externally
  and internally

  Jakob Nielsen.  Start of next section should be rewritten
  }

\section{Navigational structure should not be maintained by the author}

\textsf{reference?}  Almost all dynamic web-sites which add and delete
web pages run into the problem of maintaining site-integrity,
regarding ensuring that all the ``links'' point to the correct
document.    For external document all you can do is to check
occasionally that the page is still there, but for internal documents
the webmaster has to do the updates manually.  Updating HTML-documents
manually is at best a tedious pain, because it is hard, repetitive,
mindless labour, which should be left to a computer.

\myimage{gr/flug-1}{The Fyns GNU/Linux User Group web-site.  The side
  bar and the top graphic was added automatically by a script by
  Tobias Bardino}{flug-1}

Figure~\vref{fig:flug-1} shows a sample page from the Fyns GNU/Linux
User Group web-site, showing the way FLUG chose to do it.  The
framework shown is added to the HTML-file with a small perl script
which must be run to create each page.

This is a typical static framework for a site, where every web page
basically is ``inserted'' in the home page since the set of links is
identical for all pages.  Usually this involves a link to a search
engine for the site.

In the original web servers this required that every web page was
modified to include the snippet of HTML which produced this header,
which is a relatively easy task with a modern scripting language like
Perl.  Note that care must be taken not to change the modification
times of the files, since failing to do so invalidates local copies in
browsers and webcaches, even though that the contents of the pages
\textsf{is} unchanged.  (Modern webservers allow a webpage to include
other files with
\myurl{http://www.apache.org/docs/mod/mod\_include.html}{Server Side
  Includes}, which takes care of all this - except doing it
unconditionally on every page.  The user must still remember to insert
the appropriate command sequence in the server).

\myimage{gr/intel-1}{The web page for drivers for the Intel VS440FX
  motherboard}{intel-1}%
%
Many modern websites put a given page in a \textit{context} where the
navigational framework reflect this context.  See
figure\vref{fig:intel-1} for a sample from
\myurl{http://support.intel.com/support/motherboards/desktop/VS440fx/software.htm}{Intel
  regarding the VS440FX motherboard}, where the actual web page has a
top bar with generic links, a column with links specific to
motherboards in general, and a review form at the bottom asking
whether this information was useful to the reader (this question is
asked on every page with technical content).  My personal experience
with the Intel site is that their search engine is efficient, but that
the link column is too uninformative - you often have to actually
follow the link to see what is there.  An expanded view would be nice.

This site layout require a bit more discipline for the web author
writing each page, but is still easily implementable since it can
still be implemented by including the appropriate HTML-snippet, as
these are the same for all pages regarding this motherboard.  A way to
easily ensure this, would be to let the name of the physical file
determine the virtual location in the web server hierachi - the shown
file could have been named
\texttt{motherboards/vs440fx/drivers.html}.  This may be difficult to
manage (if the webmasters work on different machines) and it is not
very easy to change if the global layout of the web server changes.

In order to ensure integrity in the navigational framework for a site
this size, it should be created automatically.  Doing so requires
meta-data about the individual pages in order to place them correct in
the framework.

\section{A good website needs meta-information about its documents}

What to a webmaster is a nice, and well maintained web-site, is to a
computer just a bunch of directories of files with bytes in them.  In
order to get any use of a computer in maintaining these, it is
important to have easily accessible information about the desired
functionality and the files to work with.  Even though complicated
rules can be constructed to extract information from webpages, the
basic rule is still that

\begin{center}
  \fbox{\textit{A human must enter the basic
  information for categorizing a given webdocument}. }
\end{center}

(Computers are simply not good enough yet to guess this themselves).

\textsf{check spelling of altavista} Incidentially this is also the
reason why Internet search engines like AltaVista needs elaborate
information extraction techniques in order to remain useful.
AltaVista (which is discussed in section~\vref{sec:alta-vista}) was
the first Internet Search engine to cover \textit{all} webpages.  It
was immediately a great success since the Internet had already grown
to a size where help was essential to find any page if you didn't have
a direct URL to it already.

\textsf{Check precise way to do it and expand the text a bit.  Web
  pages were considered in isolation or compared with the rest of the
  site?}

It didn't take long for web authors (especially those with adult
material) to realize that the best way to get their web pages
frequently returned in a top position at AltaVista was by putting as
many potential search terms in META-tags in each and every of their
web pages, showing that just blindly extracting keywords
\textsf{uncrititically} is too simple a method.  Additionally this
drives users away -- when they feel that they are not getting ``the
best results'' from the search engine they will look for another one
that can.  The Google search engine (see~\vref{sec:google}) was such
an engine, and users quickly started to use it. Recently links to
Amazon.com and Fatbrain.com have begun to crop up in the top of the
search results on Google, and as a direct result users are going
elsewhere.  I have been recommended
\myurl{http://www.alltheweb.com}{AllTheWeb} as a good alternative. 


AltaVista failed then, because they had no control over authors and
the authors had a desire for ``breaking the system''.  For further
reading, Douglas Hofstadter talks a lot about the impossibility of
building an unbreakable system for automatic detection of bad input
in~\cite{goedelescherbach}.

\myimage{gr/mip-1}{The MIP Recently Changed Pages.  The last of the
  system pages and the first user pages are shown.}{mip-1}
An example of a document generated from meta-data which has been
automatically extracted, is the MIP ``Recently Changed Pages'' (see
figure~\vref{fig:mip-1}) the original version of
which I wrote while working for MIP.  The
script traverses three disjunkt set of web pages on the server -
System pages, Sysop pages and User pages (one set per user) - and
generates a list for each set sorted by title.  Each file listed has
its age in days next to it.

The meta-information extracted was:

\begin{itemize}
\item \textbf{Title} - used for the link, and sorting the entries in a
  set.  Extracted from the content of each document, with a default of
  the filename if no title was present.
\item \textbf{Age} - extracted from the underlying filesystem which
  registers the last change of the document
\item \textbf{User} (for the user pages) - also extracted from the
  filesystem (\textsf{or was it from the URL?}).  It is also used to
  look up the picture of the user.
\item \textbf{Size} - \textsf{Did I use it?  For anything?  Check code}
\end{itemize}

To me this is just about all the useful information a flat
Unix-filesystem can provide.  Additionally nothing at all is
guaranteed about the contents of a HTML-file - even the title is not
even always there! 

If more meta-data than the above is needed, the authors must be
involved and as a part of their web-authoring, deliberately and
carefully ensure that the meta-information needed by the automatic
processes is correct and up-to-date.

\framebox{\textsf{Is there an Internet Library effort out there?}}


The \textit{Yggdrasil system} (see section~\vref{sec:yggdrasil}) was
my first attempt to generate a navigational framework from information
extracted from webpages.  Yggdrasil was to function as the automatic
webmaster on the intranet, in order to avoid having to assign staff to
do so.  Then the individual employee could publish information in form
of a webpage, add a category either in the title or as a META-tag, and
trigger the next update of the framework.  This update would scan all
web-pages, and extract tuples of (author, category code, publishing
date, title, size) of those web-pages which had a category code, and
generate a tree structure of web-pages to navigate the categories.
Additionally lists of ``Documents sorted by author'' and ``Documents
sorted by date'' were generated to help users locate documents.

This was on a closed Intranet, where the users were interested in
using this as a tool.  The incentive for ``breaking the system'' was
very low, and the tool worked well.

Yggdrasil worked very well initially especially when its very low
amount of meta-information is taken in consideration.

\myimage{gr/yahoo-1}{The British Yahoo site}{yahoo-1}An example of a
good site built with meta-information is
\myurl{http://www.yahoo.com}{Yahoo} (see figure~\vref{fig:yahoo-1}
which started as a directory over web pages where the maintainers
added meta-data to a lot of web-sites by categorizing them manually,
and use this information to regularily generate static navigational
pages.  At a time Yahoo really suffered by a lot of broken links as
the meta-data was not maintained, but today Yahoo usually links to a
page that \textit{is} there.

\framebox{Internet Librarians?  A comment?}

\section{Avoid chaos by keeping information together}

Given there is a need for meta-data about the documents, the question
is then \textit{where} should the authors put the meta-data?


In the programming world, a very visible example of meta-data is the
documentation for programs.  Experience has shown that it is
notoriously hard for programmers to keep the documentation up-to-date,
since it is usually considered a part of the coding process that is
not really necessary and definitively not felt to be a part of the
``creative, fun'' process of writing programs!  If the writing of
documentation was well established as an integrated part of program
development, and the programming language itself helped as much as it
possibly could, it would be easier for the programmers to keep the
documentation synchronized with the code.

Experience has shown (\framebox{\textsf{references man month}}) that
documentation should be as close to the thing it documents as
possible.  For programs, this means that the documentation should be
\textit{in the same file} as the code.  Comments are usually just that
-- brief comments -- and are often meant to explain \textit{what} the
code does instead of \textit{why}?  The abstraction level is not high
enough.



\framepage{15cm}{\textsf{grow terse - does not stand on its own - hard
    to maintain even for the author - abstraction layer: the idea was
    to \textit{hide} the code and just show the meta-data}}

Andrew S. Tanenbaum writes excellent books about Computer Science, and
his Operating Systems book [\cite{tanenbaumoperatingsystems}] contains
the complete source listing with line numbers of his Minix operating
system, along with a cross-reference of all identifiers.  That was the
best the printing industry could do in 1987 (and is typical for
several other printed versions of source code), and that was not good
enough for teaching.  It is very hard to get a grasp of what the code
does, without long and careful studies, and much flapping back and
forth.  My guess is that today Tanenbaum would write a hyperlinked
book readable in a browser, perhaps even using the Literate
Programming techniques discussed in
section~\vref{sec:liteate-programming}.

The great news is that the industry recognizes the usefulness of
documentation on the web, and the need for programmers themselves to
create such documentation.  In order to gain wide acceptance such
meta-data management systems must be \textit{standards}, either as
\textit{de-facto standards} or defined by a standards body like ANSI
or W3C.

The rest of this section detours into other areas of computer science,
where having several views on the data in question has proven
beneficial.

\subsection{Javadoc - embedding web-information in programs}
\label{sec:javadoc}

\myimage{gr/acme-1}{The ACME documentation of \textsf{??}}{acme-1}

This is perhaps the most visible and generally available meta-data
tool for Java programmers today.
\myurl{http://java.sun.com/products/jdk/1.2/docs/tooldocs/solaris/javadoc.html}{JavaDoc}
is a tool that creates documentation in form of HTML pages from Java
source code with embedded comments, where all definitions are parsed
and hyperlinked to give a full overview of the Java source code in
question.  See figure~\vref{fig:acme-1} for a JavaDoc rendered source
code at \myurl{http://www.acme.com}{ACME labs.}

JavaDoc is a specialized tool which is only suitable for generating
web-pages from Java source code, but as Sun
\myurl{http://java.sun.com/products/jdk/1.2/docs/api/overview-summary.html}{use
  JavaDoc for their reference documentation} as well as ship it with
every copy of Java Development Kit, it is a tool which is widely
available.  Programmers who need to document their code, will most
likely use JavaDoc to do so.

Javadoc has also raised the expectations of the programmers, since
they have become used to hyper-linked documentation in a browser to
accompany any code they are to use.  This tendency is a good step in
the right direction.

Others recognize this too.  On March 13, 2000 a
\myurl{http://relativity.yi.org/WebSite/opensource-javadoc/}{an open
  letter pleading for the release of JavaDoc as OpenSource} was posted
to the Internet.  They argued amongst other things that a number of
bugs needed to be fixed, and new techology like XML/XSL should be
employed too.  Hopefully this will influence Sun to help the
developers as much as possible by opening up this source too.


\subsection{The Plain Old Documentation format for Perl}
\label{sec:perlpod}

The
\myurl{http://www.cpan.org/doc/manual/html/pod/perlpod.html}{perlpod}
format was designed to be simple to write and easy to use in Perl
programs, and the overwhelming amount of documentation for Perl
confirms that this goal was reached.  This format allows you  --
with very elementary markup -- to put documentation and code in the
same file, and you may leave out either one.

All the info-files at
\myurl{http://www.unixsnedkeren.dk}{Unixsnedkeren.dk} (the website for
my small firm) have been written as POD files, and converted to HTML,
with some finishing touches done with another Perlscript modified for
the one I wrote for
\myurl{http://www.fido.dk/faq/unix-faq/unix\_r23.htm}{the FAQ for the
  Unix echo in the Danish Fidonet}.  Code truly lives forever.

Due to the \textit{ad-hoc} code in both \unixcommand{pod2html} and my
own code, I am convinced that another approach to rendering POD-files
should be taken, and I have submitted patches to the
\unixcommand{pod2docbook} command in order to create DocBook XML in
addition to the current DocBook SGML.  


\subsection{Literate Programming - Knuths approach to different views
  of the source}
\label{sec:literate-programming}

\mycitation{I believe that the time is ripe for significantly better
  documentation of programs, and that we can best achieve this by
  considering programs to be works of literature. Hence, my title:
  "Literate Programming." Let us change our traditional attitude to
  the construction of programs: Instead of imagining that our main
  task is to instruct a *computer* what to do, let us concentrate
  rather on explaining to *human beings* what we want a computer to
  do.}%
{Donald Knuth}%
{\cite{knuthliterateprogramming} quoted from
  \myurl{http://www.literateprogramming.com/}{The Literate Programming
    web site}}

Donald E. Knuth has written the {\TeX}-system used to typeset this
report, and documented it by publishing the source code to the entire
system in four books.  The {\TeX}-system has impressed since it is of
an extremely high quality, both in code and documentation, and
\myurl{http://truetex.com/knuthchk.htm}{actually offer money to those
  who find errors in his code}.

\textsf{references to the TeX book, MetaFont book, Literate
  Programming Book, nuweb (my observations)}

In order to write both documentation and code of such high quality,
Knuth developed his own method of coding \textsf{which he named
``Literate Programming''}, in which the author works with a single
file containing the documentation \textit{as it is to be presented to
the reader} written in a {\TeX}-dialect, with the actual code (with a
few characters escaped) listed in named chunks along with their
documentation, as it fits the author to present them.  This file
format is the \texttt{web}-format!

\begin{itemize}
\item 
The \texttt{weave} tool converts the web-file to a \texttt{tex}-file
to be processed by {\TeX}, and printed.

\item The \texttt{tangle} tool generates the actual code to be
compiled, by joining chunks with the same name, and inserting them in
other chunks that reference them (a simple macro expansion),
eventually producing one or more flat files which can be compiled.

\end{itemize}

\textsf{We definitively needs a sample of this!}

\textsf{Reference to literateprograming.org}

The problem with Knuth is that things that work well for him, does not
work quite as well for the rest of us.   No editors - not even Emacs -
can help where a given file is separated in a lot of regions, being 
either {\TeX}-code or source code, meaning that the editor cannot
provide the supporting functions which a programmer might have become
accustomed to.

You cannot code verbatim - some characters are reserved for other
purposes and must be written differently, which may be very annoying
to learn.  Additionally the concept of chunks all over the document
may be counter-productive if these are hard to navigate.

\textsf{What else}

\textsf{Many tried this- cloned the functionality -etc - I tried it,
  and wrote a few programs with it.}

Many people have experimented with the possibilities of a
weave/tangle pair resulting in several software packages, notably:

\myurl{http://www.ross.net/funnelweb/}{FunnelWeb},\myurl{http://www.eecs.harvard.edu/~nr/noweb/}{noweb}

\textit{http://w3.pppl.gov/~krommes/fweb.html\#SEC3}

There is
\myurl{http://www.webring.org/cgi-bin/webring?ring=litprog;list}{a
  webring for Literate Programming}, which is an excellent place to
look for further information.  The Collection of Computer Science
Bibliographies provides
\myurl{http://liinwww.ira.uka.de/searchbib/SE/litprog}{a search engine
  in literate programming publications}.

Oasis has \myurl{http://www.oasis-open.org/cover/xmlLitProg.html}{a
  page on Literate Programing with XML and SGML} \textsf{what about
  it? - note the references for working with the TEI DTD}

My previous experience with Literate Programs can be summarized as:

\begin{itemize}
\item The documentation gets bigger and better, simply because the
printed documents look better that way.  What would pass as a single
line comment in a source file, looks almost pathetic when typeset.
The full power of {\TeX} also encourages usages of illustrations and
graphs.  

\item The program development gets cumbersome.  You may have trouble
  using your favorite tools for editing, compiling and debugging.
  Users of most integrated development environments cannot use this
  model since the IDE does not provide hooks to provide this processing.
  
\item A critical point is whether the intermediate files are visible
  to the author! In order to be usable, \textit{all} derived files
  \textit{must} refer to the original document in a transparent
  fashion, meaning that the user should not have to worry about
  intermediate files.  (In the same way that the C-processor works
  under Unix).
\end{itemize}

My conclusion was that the Literate Programming paradigm does not pay
off as a individual programmer, but may work very well for a larger
programming team where good, current documentation is critical.

\subsection{Rational Rose - the other way around}
\label{sec:rational-rose}

A software product which has been very successful in recent years, is
the \myurl{http://www.rational.com/products/rose/index.jtmpl}{Rational
  Rose visual modeling tool} which is used by several people at MIP to
do software development.

Rational Rose provides for a lot of things relevant to a large
software project like reverse engineering of existing code, but also
for
\myurl{http://www.rational.com/sitewide/support/whitepapers/dynamic.jtmpl?doc\_key=350}{several
  views of the document} depending on what abstraction level the
authors are working on.  In this way they are able to provide the
author an environment where the software model can be designed
seperately from the writing of the code.

Bo N�rreg�rd J�rgensen told me that he would expect it to be
beneficial to use Rational Rose in a software project when it has an
underlying model and has 7 modules or more.

\subsection{Compiling source code into an executable}
\label{sec:compiling-source-into-an-executable}

Most software projects known to me does not consist of a single huge
source file which would take ages to compile, but of several smaller
files, which can be compiled individually and linked to an executable
file.  If a file with source code is changed, it is not necessary to
recompile each and every file but only those which depend on this
particular source file, and then relink the executable to incorporate
the changes.

This is possible because the authors has provided meta-data about the
system, regarding which source files the system contains, which
commands to call to compile and link the source files, which libraries
should be included, etc.  Normally an Integrated Development
Environment (like \myurl{http://www.borland.com/bcppbuilder/}{Borland
  C++Builder}, \myurl{http://msdn.microsoft.com/visualc/}{Microsoft
  Visual C++}, or \myurl{http://www-4.ibm.com/software/ad/}{the Visual
  Age products from IBM}) knows about these things, or a
\myurl{http://www.eng.hawaii.edu/Tutor/Make/}{Makefile} is constructed
to utilize the known relations built into \unixcommand{make} as well
as allow the author to add new ones.

The real advantage comes in environments (multi-threaded, multi-CPU)
where several compilations can run in parallel.  This can be done
safely by using meta-data to localize compilations which are
independent from one another and execute these simultaneously.  MIP
have had a 24-CPU Silicon Graphics machine in which using a
parallelizing ``make'' could reduce compilation times with a factor of
20.   Not all software projects were easily parallelized - if the
Makefile did not list all dependencies explicitely but expected that
the normal compilation order would produce the unlisted files, the
parallelized compilation would fail.

\medskip

In effect, adding meta-data to the project allow the program
development process to go faster.


\section{Requiring multiple views of a document}
\label{sec:requiring-multiple-views-of-a-document}

\myimage{gr/zdnet-1}{The ZDNet presentation of a story -- notice how
  little of the initial screen that is dedicated to the story about
  ``Microsoft may try to boost WinCE, Linux-style''}{zdnet-1}
\myimage{gr/zdnet-2}{The ``Printer-friendly version'' of
  figure~\vref{fig:zdnet-1}}{zdnet-2}

Some web-sites have so many advertisements and auxiliary links in
their layout, that they need an additional version which is
well-suited for printing (i.e. only has a single ad).  See
figure~\vref{fig:zdnet-1}
\myurl{http://www.zdnet.com/zdnn/stories/news/0,4586,2468874,00.html?chkpt=zdhpnews01}{for
  an example from ZDNet}, and figure~\vref{fig:zdnet-2} for the
``Printer-friendly version''.

It is interesting to notice that websites realize that they cannot
just offer an advertisement ladden service, without allowing for a
reasonable paper version.  Since their HTML cannot ``remove'' the ads
for printing, they have to offer two different version of each
article.

\myimage{gr/nokia-1}{Browsing the Internet on the large [sic] display
  of a Nokia 9110}{nokia-1}

These are two different \textit{views} of the same basic document.
ZDNet are expecting their visitors to either see their documents on a
modern display unit with high resolution and millions of colors
rendered by a reasonably capable browser or print them out on paper.
That will probably change within a few years when it becomes common
for other devices than just traditional computers to be connected to
the Internet.  Mobile phones using WAP do not have as much screen real
estate (the Nokia 9110 is shown in figure~\vref{fig:nokia-1}) so such
users will probably prefer short and crisp versions of the documents
when they use a mobile phone.  Nokia is working with 3com to produce a
hybrid between a Palm Pilot and a mobile phone.

This tendency might as a side-effect be beneficial for the currently
rather overlooked blind computer users (see
\myurl{http://www.webring.org/cgi-bin/webring?ring=blind\&list}{the
  Blind Web-ring} for more information) .  They must normally use
text-to-braille software or a ``screen reader'' in order to use
computers, neither of which work well with web pages loaded with
graphics, but when web authors must write for many kinds of media
instead of just a particular version of a given browser, this will
also be beneficial to these users since it will be easy to write a
renderer for their preferred format.  Of course, it is possible to
write ordinary HTML in a way that is usable by these users, but that
is rare these days.

\framebox{A reference to the Jakob Nielsen site on web-accessability}

\section{SGML/XML:  Generic formats for storing data and meta-data}
\label{sec:sgml-and-xml-generic-formats}

The problem of representing content in a generic way is not new at
all, and was solved for the printing industry in the 1970'ies by the
design of the Standard Generalized Markup Language (SGML) which is
widely used.  HTML and XML are both SGML-dialects abeit with different
capabilities.  Chapter~\vref{cha:sgml-xml-and-dtd's} talks a lot more
about this.

For now it is sufficient to say that it is possible to store both data
and meta-data in a convenient form in a SGML or XML file, but that
these files must be rendered into the formats expected by the users,
like HTML, PDF or what else tomorrow may bring of new, exiting
formats.  Since SGML describes the \textit{content} and not the
layout, it is ``just'' a matter of creating a renderer for any new
format and add it to the collection.

This has been recognized by almost all of the ``Documentation
Projects'' (\myurl{http://www.linuxdoc.org/}{The Linux Documentation
  Project}, \myurl{http://www.freebsd.org/docproj/docproj.html}{The
  FreeBSD Documentation Project} and others) accompanying the various
free operating systems available on the Internet.


Chapter~\vref{cha:on-demand-rendering} talks about rendering documents
to a given format on demand (for example in a web browser).

\framebox{Should I talk a little bit about O'Reilly?}

% \framepage{15cm}{
% Introduce publishing where HTML is just a backend amongst many (PDF,
% Word, ASCII).

% Describe why it is important to be able to render finished versions of
% documents fully automatically from a single \textsl{annotated} source
% (www/print/cdrom/Palm Pilot/braille/handicapped persons,whatever).  The better
% the annotation, the better the output (ref: Stibo).  Describe the need
% for SGML (history/usage) and XML (why/browser support/on-the-fly
% publishing/bleeding edge - being standardized).

% Donald Knuth - {\TeX}/Web/Literate Programming - advantages (high
% quality, excellent math, superb algorithms, can be tailored to needs
% (basic interpreter written in TeX) and
% disadvantages (programming language, not abstract) (designed 20 years ago).  Ask Steffen Enni
% about his thoughts.  Javadoc.  Perl POD.  DocBook projects (FreeBSD,
% LDP).  
% }

\section{The user should use familiar tools to publish documents}
\label{sec:the-user-should-use-familiar-tools-to-publish-documents}

\framepage{15cm}{
SGML-editing is hard and tedious - this is one thing that the folks in
the \myurl{news:comp.text.sgml}{\texttt{comp.text.sgml}} and
\myurl{news:comp.text.xml}{\texttt{comp.text.xml}} newsgroups agree
upon.  Tools are essential for authors.

Very few tools exist, and the good ones are quite expensive.  The
general consensus is that the best OpenSource tool is the Emacs editor
with the PSGML package (see~\vref{sec:emacs-with-psgml}), and that
this tool is primarily suited for programmers and other people who are
well acquainted with SGML and XML.

\textsf{PRINT DIRECTLY TO A PDF FILE - WEB PUBLISHING}

Computers should \textit{help} you do your work, and users are
generally most productive with the tools they know.  Why should an
author use SGML with an editor she loathes, if almost the same result
can be achieved by using a word processor she is familiar with?

In order to automate the conversion from e.g. Word to the equivalent
SGML file, it is important that the author uses only those
constructions that the conversion program understands.  This is most
likely given as a set of macros that must be used, as well as a
verification program which the author can use at any time to check
whether her document is conformant.

Publishing a document to the web should be as easy as printing a
document.  The basic principles are the same - you can do with a
``Publish''-button and a dialogue box where the essential information
is provided.  It is not so today - discuss reasons .  Use ``print to
fax'' software as horrible example of this idea gone wrong.

When the webserver is dumb, you cannot do much.  If there is a
database underneath, much more is possible.  Discuss the idea of
having several ways of entering documents in the database (upload via
form, send as email (fax software can do this too), print to virtual
printer, scan, fax, voice) and letting the software do the
conversions.

Use example with print PostScript to file and upload via ftp to
printer (LexMark).

Writing XML directly is much harder to do than writing HTML (stricter
syntax, more options, plainly just more to type), and should be aided
by a good tool.  Alternatively, the user should use well-known tools
(Word) and mark up according to very strict rules, which is then
automatically converted to the XML document.  This does not give as
rich documents, but allows users to publish existing documents with
very little trouble.
}

%%% Local Variables: 
%%% mode: latex
%%% TeX-master: "rapport"
%%% End: 

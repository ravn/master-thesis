\chapter{Conclusion}

This thesis has discussed many topics regarding the current state of
databases plus web publishing, and has demonstrated a new approach to
integrating these tools in a normal working environment while
requiring very little of the end-user, both in terms of software but
also the effort in learning new programs and ways to work.

\subsection*{Implementation languages}

Even though I have not implemented Cactus fully, I am very satisfied
with it, especially regarding the ease with which an application
programmer can implement new filters, and the flexibility and power
that is allowed in invoking these filters under Unix.  It is possible
to develop and test filters outside of Cactus, and first moving inside
when the individual parts of the filter have been tested and tried.

The MySQL support in both Perl, PHP3 and Java have allowed me to
choose freely amount all three platforms when choosing a language for
a given task, giving a freedom of how to do things that has been very,
very nice, as well as giving a lot of opportunities, especially with
Java.

The Cocoon technology allowing on-the-fly rendering of XML-documents
to HTML has been fun to work with -- especially with the ability to do
SQL-requests directly in the XML-file -- as well as being an
intellectual challenge.  The XML and XSTL concepts are easy to work
with, but I found the XPath language for doing expressions extremely
nasty and \textit{very} hard to learn properly.



\subsection*{Underlying database}

The impact of building Cactus around a database has truly been
\textit{incredible}, and has been the area in which I have learned the
most, and even though the learning curve has been much steeper than
anticipated, this is much overshadowed by the ease of which new things
can be implemented -- the search engine took about 5 minutes -- and
the amount of tedious administration which is handled transparently by
the database.  Several programs running on multiple computers can do
work without having to implement elaborate synchronization mechanisms,
thus providing a design which is easy to extend to larger systems.

The \textit{really} hard part was designing the database tables, and
realising that a more complex model was required than just thinking
``this is a file''.  The ability to script database maintenance
meant it was easy to rebuild the database from scratch whenever the
basic design changed.  

I have regretted the actual \textit{choice} of database.  MySQL has a
limitation that I have had to work around, namely the lack of
sub-selects which means that derivers and data gatherers needs to
implement the extract, do work, error check, and reinsert result
themselves making them unnecessarily complex, which should properly be
placed in the database.   A more suitable database for this purpose
should be found, and the scripts adapted to work with it.


\subsection*{The future of Cactus}

My original intent was to have Cactus running for several months --
preferably half a year -- to see how people would use it, get feedback
and improve on it.  This I didn't do, primarily because it took much
longer than expected to learn SQL well enough to have things work.  I
do not expect the current installation to live long on MIP, but I
intend to continue development into a more full-fledged software
package, where other developers can help.

It was demonstrated in \myvref{sec:xml-publishing} that Cactus can be
used to automatically generate web documents in many formats from a
single XML-source file, which was sent to the system as an email as
the only publishing action the author had to do.

It is my strong belief that this is the way that web publishing should
work -- trouble-free to publish, and trouble-free to view.  I hope
that my work can help in achieving that goal.




\begin{center}

\vspace{2cm}
Odense 2000-05-07

\vspace{2cm}
Thorbj{\o}rn Ravn Andersen

\end{center}


%%% Local Variables: 
%%% mode: latex
%%% TeX-master: "rapport"
%%% End: 

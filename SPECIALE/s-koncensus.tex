% $Id$

\chapter{\textit{Konsensus} -- an unimplemented collaborative OpenFAQ system}
\label{cha:konsensus}

A few years back I authored and maintained
\myurl{http://www.fido.dk/faq}{a Unix FAQ in the Danish Fidonet},
where I collected a lot of information and wrote a lot too myself,
which I then formatted and made available to the community for the
common benefit.

The typical problem with a FAQ is that the maintainer must be quick to
update with corrections when users take the time to submit them, and
be certain to credit the users (to get the incentive for them to see
their names listed in a community resource).  The maintainer must also
be sure to review the contents often to ensure that the information is
up-to-date and correct.

This is a non-trivial task, and after basically giving up on the third
rewriting of the UNIX-FAQ about halfway through, I concluded that for
a FAQ to be truly successful a number of criteria must be met:

\begin{itemize}

\item \emph{The information must be current and correct}.  This is
  what makes people use the resource in the first place.
  
\item 
  \emph{The bottleneck in this system is the maintainer}.  The system
  should allow users to help you with updating the information in the
  FAQ!

  
\item \emph{The person providing the information must be given
    credit}.  Credit and reputation is the mechanism that gets readers
  to submit updates and new information.  Eric S. Raymond writes about
  \myurl{http://www.tuxedo.org/~esr/writings/homesteading/homesteading-8.html}{the
    importance of reputation in the hacker gift culture} in the paper
  ``Homesteading the Noosphere''.
\end{itemize}

I was not the only one to reach such conclusions.  Thomas Boutell
developed the \myurl{http://www.boutell.com/openfaq/source}{The
  OpenFAQ system} which divided the FAQ into a page per question with
the corresponding answer.  If a reader was better informed regarding
the answer, a link was provided to a fill-out form where the updated
information could be entered.  The finished form was then saved in the
system, and an email sent to the FAQ maintainer so that he could check
that the new text was suitable for updating the old text.  It was the
maintainers responsibility to maintain the questions.

Even though this approach gives the maintainer an excellent method of
organizing responses from users, it does not help much in the
viewpoint of the submitting reader if the maintainer takes too long to
accept the submission.  It took three weeks for my personal
submission to be accepted, which was plainly too long.

It is very understandable that the moderator wants to see the new
submissions before activating them, in order to avoid electronic
grafitti and vandalism, but that requires a high degree of visibility
of the moderator.  The site is not capable of running without
supervision.

The answer to this is to give the users access to post (after they
have identified themselves) \textit{and} then deal with malicious
posters when it happens.  It appears that requiring the user to
specify a valid email address, and get login information by email, is
sufficient to weed out most of the vandals.

Recognizing this I outlined the ``Konsensus'' system which was
intended to work in combination with a usenet autoposter for one or
more groups.  I have been involved in starting the autoposter system
in the Danish newsgroup, where a given newsgroup is under surveillance
by a daemon that looks at the email address in the ``From''-field and
checks with a database whether that address is ``new'' (has posted
within the last 3 months or so).  If not, an email is sent to that
address with a canned message (like the
\myurl{http://www.usenet.dk/oss/dk.edb.programmering.perl/intro.txt}{``dk.edb.programmering.perl''
  message} which I wrote (the Perl group has many new Perl programmers
since many Internet Service Providers provide Perl as their
CGI-scripting language).  This proved to be extremely efficient, since
the regular posters \textit{knew} that the newcomers would be greeted
with a welcoming post containing the information they needed, and
efforts could be saved for dealing with more complicated issues.

Such a welcoming message suffers from exactly the same problem as a
FAQ, namely that it requires regular attention in order to be as
useful as intended by the author, so I decided there was a need for a
system which would allow the users of the newsgroup to collaborate on
making the welcoming message as useful as possible until they reached
a consensus (hence the name).

Basically this would involve the following:

\begin{itemize}
\item The maintainer were responsible for creating, modifying and
  deleting the questions -- as these were intended to be rather static
  -- after discussing this either in the news group or via email
\item The users were responsible for changing the content of each
  page, by modifying the HTML-source of the answer part of each
  question.
  
\item Each modification would be checked in to RCS, allowing any
  reader to \textit{go back in time} to see what earlier answers held
  instead of just having the latest edition which might not be 100
  percent correct.  These editions would be labeled with the email
  address of the latest editor.
  
\item The web pages would be generated automatically whenever the
  underlying files were changed.
  
\end{itemize}

This were at the time where I was evaluating which language I should
use for implementing Cactus (see section~\vref{sec:cactus}) so I
followed both the Java group and the Perl group carefully while doing
so, which in particular prompted me to look for such software.  The
problem is that in the same way that the Perl groups are bothered by
new programmers which ask questions about webservers (and not about
perl) and therefore belongs in another group, the Java groups are
bothered by new programmers which ask questions about JavaScript which
is considered HTML-programming which also belongs in another group.

To make a long story short, I then discovered the
\myurl{http://www.dartmouth.edu/~jonh/ff-serve/cache/1.html}{FAQ-O-Matic}
which does most of what I wanted the Konsensus system to do (except
for the RCS storage with credit), and which runs as a set of CGI
scripts.   I installed it under Linux, played with it for a while, and
found it to be  quite suitable for the job.

As I then had decided that Perl would be the best language for the
daemon part of Cactus due to the library support, I concentrated on
that from that point.  The Konsensus system can be quite easily
implemented for a given newsgroup with a FAQ-O-Matic combined with a
usenet harvester I wrote for Cactus, but I have not actually done so,
since the Cactus project was much more interesting.



  
%%% Local Variables: 
%%% mode: latex
%%% TeX-master: "rapport"
%%% End: 

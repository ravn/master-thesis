% $Id$

\chapter{Report writing tools}

When writing a large document like a thesis, you get to appreciate the
power of the tools available to you in the Open Source community.
This chapter outlines which tools I have used while writing and my
experiences with them, with the thought that they might be useful to
others later, and Open Source is all about not reinventing the wheel
if it can be avoided.  

All the software listed here, is available with a standard Linux
Redhat 6.1 installation unless noted otherwise in the text.


\section{\LaTeX}
\label{sec:report-writing-tools-latex}


Guide is~\cite{a-guide-to-latex}.


\subsection{Creating PDF files}
\label{sec:report-writing-creating-pdf-files}

I have had good success with the following approach to creating
PDF-files from a \LaTeX-file.

\begin{itemize}
\item Use a PostScript font!  Many \LaTeX-installations use the
  standard \TeX-font 'Computer Modern' as a bitmap font, which doesn't
  look good as a PDF-file at other resolutions.   Try
  \texttt{\texcommand{usepackage}\{times\}} or
  \texttt{\texcommand{usepackage}\{bookman\}} to see if that helps.
\item Use ``\unixcommand{pdflatex}'' to generate the PDF-file
  directly.  Included graphics may have to be converted to
  \texttt{png} instead of \texttt{eps}.  The traditional approach with
  \unixcommand{dvips} and \unixcommand{pstopdf} \textsf{from
    Ghostscript}
\item Develop the document to work with both \unixcommand{latex} and
  \unixcommand{pdflatex}.  The edit-compile-view cycle is much faster
  with \unixcommand{latex} and \unixcommand{xdvi} than with
  \unixcommand{pdflatex} and \unixcommand{acroread}.
\end{itemize}

I have had problems with printing the PostScript file from
\unixcommand{dvips} directly to some HP-printers.  Results have been
much better with printing from
\myurl{http://www.adobe.com/products/acrobat/readstep.html}{Acrobat
  Reader} (can be done directly from the command-line with
\unixcommand{cat source.pdf | acroread -toPostScript | lpr} -- use the
\texttt{-helpall} option to get a full overview).


\subsection{Including screen dumps}
\label{sec:report-writing-tools-latex-eps}

Screen dumps have been captured with
\myurl{http://www.freshmeat.net/appindex/1998/07/28/901622941.html}{xv}
(which is now 6 years old and still unbeat) and saved as \texttt{tiff}
files.  These were then converted to \texttt{png} with
\unixcommand{pnmtopng} and to \texttt{eps} with the
\myurl{http://www.freshmeat.net/appindex/1999/06/24/930238715.html}{NetPBM
  package}, which has worked flawlessly.  It is important to remember
to use \texttt{pnmtops -noturn rle} to get the correct orientation and
a reasonable compression (25 files shrank from 52 Mb to 9 Mb total).

The \texcommand{includegraphic} (from the \texttt{graphicx} package)
can read \texttt{png}-files from within \texttt{pdflatex}, and
\texttt{eps}-files from within \texttt{latex}


The process of embedding Postscript is fully described in
~\cite{the-latex-graphics-companion}.

% \begin{verbatim}
% [ravn@personal-53 gr]$ cat *.eps | wc
%  872378  873482 53187873
% [ravn@personal-53 gr]$ rm *.eps
% [ravn@personal-53 gr]$ make -s
% [ravn@personal-53 gr]$ cat *.eps | wc
%  149509  151878 9063019
% [ravn@personal-53 gr]$
% \end{verbatim} %$

I have had ``\texttt{xdvi}'' active while writing, and if the command
``\texttt{latex mydocument \&\& killall -USR1 xdvi}'' is used, xdvi
will automatically update after a successful compilation of the
TeX-document.  That is very nice when your editing have happened
within a single physical page.

For presentation purposes, \textsf{Acroread with packages found at the
  DK-TUG website at aucdk}

\subsection{Finding back to the source document from xdvi}

When reading the typeset document (either in \unixcommand{xdvi},
\unixcommand{acroread} or on paper) it is often a problem to find the
appropriate \texttt{tex}-file.  I used this trick which was deduced
from the \textsf{appropriate latex book here Daly?}, which puts the
filename to the left in the running header on each page.

\begin{verbatim}
\usepackage{fancyhdr}           % provide headers where I can put the filenames.
% Trick from page 16 in fancyhdr documentation
\newcommand{\currentinputfile}{}
\newcommand{\myinclude}[1]{%
  \renewcommand{\currentinputfile}{File:\texttt{#1}}\include{#1}%
  \renewcommand{\currentinputfile}{}}

\lhead{\currentinputfile}
\pagestyle{fancy}

...
\begin{document}
\myinclude{s-front-matter}
\myinclude{s-introduction}
...
\end{verbatim}

\textsf{Modified xdvik}!

\section{Emacs} 
\label{sec:report-writing-tools-emacs} 


Emacs is a very well known editor in the Unix-world where its
extensability allows it to get new functionality by installing
additional modules.  The standard distribution of Emacs provides:

\begin{description}
\item[Ispell] -- The M-Tab command (bound to
  \texttt{TeX-complete-symbol}) runs \texttt{ispell-complete-word} if
  the current word is not a {\TeX}-command, which suggest possible
  words starting with the word under point.  The M-\$ command (bound
  to \texttt{ispell-word}) is a great help when in doubt about a
  word.  The \textsf{flyspell-mode} highlights words while you type,
  akin to the wavy underline used in MS-Word. 

\item[CVS] -- Emacs supports CVS directly.  I
  had my working directory on my portable, and my CVS repository on
  the MIP system, in order for my CVS files to be automatically backed
  up. 
\item[Compile] -- The \texttt{M-x compile} works well with Jade,
  nsgml, Java and other tools where you may encounter errors in your
  source.  \texttt{Ctrl-x `} makes Emacs go to the line with the
  error.
  
\end{description}

Additionally I have installed:


\begin{description}
\item[AUC-TeX] -- provides a complete IDE for writing {\LaTeX}
  documents, including keyboard shortcuts, a debugger, and intelligent
  commands depending on the state of the files.  \textsf{url?
    somewhere at sunsite}.  Highly recommended.
\item[psgmls] -- provides easy entering of tags, and parses DTD's to
  verify tags, and describe which tags are open at a given cursor
  location.   Good for writing XML and SGML directly.
  \myurl{http://www.lysator.liu.se/projects/about\_psgml.html}{The
    psgml homepage} - \textsf{consider writing a set of AUC-TeX
    compatible keybindings for DocBook}
\end{description}



\subsection{Making them all work together}
\label{sec:making-them-all-work-together}

Using these packages has basically been very easy, but occasionally
they step on each others toes.  That could normally be circumvented:

\begin{description}
\item[{\TeX} and RCS] -- The RCS header fields, contain dollar-signs
  which causes {\TeX} to enter/leave math-mode (like
  \mbox{\$I}\mbox{d\$}), causing it to be rendered in math italic
  $which is not suitable for prose$.  By typing it as \texttt{\$
    \${}Id\$ \$} it is possible to avoid this. The dollar-space-dollar
  renders the space in math mode and switches back to normal text.


\end{description}

%%% Local Variables: 
%%% mode: latex
%%% TeX-master: "rapport"
%%% End: 

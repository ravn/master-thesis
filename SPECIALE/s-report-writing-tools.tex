% $Id$

\chapter{Report writing tools}

When writing a large document like a thesis, you get to appreciate the
power of the tools available to you in the Open Source community.
This chapter outlines which tools I have used while writing and my
experiences with them, with the thought that they might be useful to
others later, and Open Source is all about not reinventing the wheel
if it can be avoided.

All the software listed here, is available with a standard Linux
Redhat 6.1 installation unless noted otherwise in the text.


\section{\LaTeX}
\label{sec:report-writing-tools-latex}


Guide is~\cite{a-guide-to-latex}.



\subsection{Including screen dumps}
\label{sec:report-writing-tools-latex-eps}

Screen dumps have been captured with
\myurl{http://www.freshmeat.net/appindex/1998/07/28/901622941.html}{xv}
(which is now 6 years old and still unbeat) and saved as \texttt{tiff}
files.  These were then converted to \texttt{png} and \texttt{eps}
with the
\myurl{http://www.freshmeat.net/appindex/1999/06/24/930238715.html}{NetPBM
  package}, which has worked flawlessly.  It is important to remember
to use \texttt{pnmtops -noturn rle} to get the correct orientation and
a reasonable compression (25 files shrank from 52 Mb to 9 Mb total).

The \texcommand{includegraphic} (from the \texttt{graphicx} package)
can read \texttt{png}-files from within \texttt{pdflatex}, and
\texttt{eps}-files from within


The process of embedding Postscript is fully described in
~\cite{the-latex-graphics-companion}.

% \begin{verbatim}
% [ravn@personal-53 gr]$ cat *.eps | wc
%  872378  873482 53187873
% [ravn@personal-53 gr]$ rm *.eps
% [ravn@personal-53 gr]$ make -s
% [ravn@personal-53 gr]$ cat *.eps | wc
%  149509  151878 9063019
% [ravn@personal-53 gr]$
% \end{verbatim} %$

I have had ``\texttt{xdvi}'' active while writing, and if the command
``\texttt{latex mydocument \&\& killall -USR1 xdvi}'' is used, xdvi
will automatically update after a successful compilation of the
TeX-document.  That is very nice when your editing have happened
within a single physical page.

\section{Emacs} 
\label{sec:report-writing-tools-emacs} 


Emacs is a very well known editor in the Unix-world where its
extensability allows it to get new functionality by installing
additional modules.  The standard distribution of Emacs provides:

\begin{description}
\item[Ispell] -- The M-Tab command (bound to
  \texttt{TeX-complete-symbol}) runs \texttt{ispell-complete-word} if
  the current word is not a {\TeX}-command, which suggest possible
  words starting with the word under point.  The M-\$ command (bound
  to \texttt{ispell-word}) is also nice when English is not your
  native tongue.
\item[CVS] -- Emacs supports \myurl{\texttt{???}}{CVS} directly.  I
  had my working directory on my portable, and my CVS repository on
  the MIP system.  This is a good way to backup your work.
\item[Compile] -- The \texttt{M-x compile} works well with Jade,
  nsgml, Java and other tools where you may encounter errors in your
  source.  \texttt{Ctrl-x `} makes Emacs go to the line with the
  error.
  
\end{description}

Additionally I have installed:


\begin{description}
\item[AUC-TeX] -- provides a complete IDE for writing {\LaTeX}
  documents, including keyboard shortcuts, a debugger, and intelligent
  commands depending on the state of the files.  \textsf{url?
    somewhere at sunsite}.  Highly recommended.
\item[psgmls] -- provides easy entering of tags, and parses DTD's to
  verify tags, and describe which tags are open at a given cursor
  location.   Good for writing XML and SGML directly.
  \myurl{http://www.lysator.liu.se/projects/about\_psgml.html}{The
    psgml homepage} - \textsf{consider writing a set of AUC-TeX
    compatible keybindings for DocBook}
\end{description}



\section{Available databases}
\label{sec:available-databases}

The number of database vendors which support Linux (which as for now
is \textit{the} Open Source operating system which people know about)
is steadily increasing.  \myurl{http://linas.org/linux/db.html}{This
  webpage try to keep track of them all}, but I only list the major
commercial vendors and Open Source projects, where the drivers should
be of production quality.  The following overview was created March
2000.


\subsection{Cloudscape - Informix}
\label{sec:cloudscape}

The
\myurl{http://www.informix.com/informix/press/1999/dec99/cloudscape.htm}{Cloudscape
  Database} is written in Java, with a JDBC and a HTML interface, has
many core SQL and Java extensions implemented, and allow distribution
amongst several Java-machines (using RMI).  Informix bought Cloudscape
Inc. in October 1999 to get this database system, so it is still very
new.  They offer
\myurl{http://www.cloudscape.com/Evaluations/index.html}{a 60 day
  trial version from their website}.  Pricing is approximately \$900.


\subsection{DB2 - IBM}
\label{sec:db2}

\myurl{http://www-4.ibm.com/software/data/db2/linux/}{DB2 for Linux}
is available - the \myurl{DB2 Personal Developer's Edition
  V6.1}{http://www6.software.ibm.com/dl/db2pde/db2pde-p} is available
for free for non-commercial purposes.  

\myurl{http://www-4.ibm.com/software/data/db2/extenders/xmlext/}{DB2
  XML Extender} which ``let you store XML documents in DB2 databases
and new functions that assist you in working with these structured
documents.''.  It is available for AIX, Solaris and Windows NT.
  \textsf{Evaluate?} 

  

\subsection{Informix - Informix}
\label{sec:informix}

http://www.informix.com/datablades/dbmodule/informix1.htm

 
\subsection{Ingres - Ingres}
\label{sec:ingres}

The
\myurl{http://www.cai.com/products/betas/ingres\_linux/ingres\_linux.htm}{Ingress
  II database is in a beta stage for Linux}, and can be downloaded for
free.  It uses Perl, gcc and Apache to provide webserver facilities.



\subsection{MySQL - TCX}
\label{sec:mysql}

\myurl{http://www.tcx.se}{MySQL is an Open Source database} which is
very popular amongst Perl programmers due to
\myurl{http://www.tcx.se/benchmark.html}{the good performance}
combined with the free availability of source code and high quality
drivers.

MySQL only costs money if you run it on a Microsoft operating system,
or commercially on a server.

\subsection{Oracle}
\label{sec:oracle}

The complete Oracle 8i product range is available for Linux, with a
low-end version available for download.  The WebDB tool functions as a
webserver-interface to any Oracle database.

\subsection{SyBase - Sybase}

Sybase \myurl{http://www.sybase.com/products/linux/}{has Sybase
  Adaptive Server Enterprise and SQL Anywhere Studio} available for
  Linux for free as long as they are used for development.  For
  production installations a license must be bought.  An earlier
  version is unsupported but may be used without restrictions for
  production enviroments.
  
  Their web integration tool does not appear to have been ported to
  Linux yet.  The \myurl{http://linas.org/linux/db.html}{Linux
    database page} lists several third-party tools which provide web
  functionality.

  
  


%%% Local Variables: 
%%% mode: latex
%%% TeX-master: "rapport"
%%% End: 

% $Id$

\chapter{Report writing tools}

When writing a large document like a thesis, you get to appreciate the
power of the tools available to you in the Open Source community.
This chapter outlines which tools I have used while writing and my
experiences with them, with the thought that they might be useful to
others later, and Open Source is all about not reinventing the wheel
if it can be avoided.  

All the software listed here, is available with a standard Linux
Redhat 6.1 installation unless noted otherwise in the text.


\section{\LaTeX}
\label{sec:report-writing-tools-latex}


Guide is Kopka and Daly \cite{a-guide-to-latex}.  An excellent book,
which is strongly recommended to have on the desk while writing, since
it deals with both basic things and advanced stuff, where the
traditional LaTeX-books thend to concentrate on just one of these.  I
found third edition to be substantially better than the second edition.

% They recommend the \texttt{variouref} package to get pagenumbers on
% your references to figures, tables etc. but that packages does not
% work well with the hyperref package used to generate links in the
% PDF-version.  


%\subsection{Creating PDF files}
%\label{sec:report-writing-creating-pdf-files}

I have had good success with the following approach to creating
PDF-files from a \LaTeX-file

% (as opposed to the traditional
% \texttt{dvips} + \texttt{ps2pdf} sequence which usually ends up with
% bitmapped fonts).

\begin{itemize}
\item Use a PostScript font!  Many \LaTeX-installations use the
  standard \TeX-font 'Computer Modern' as a bitmap font, which doesn't
  look good as a PDF-file at other resolutions.   Try
  \texttt{\texcommand{usepackage}\{times\}} or
  \texttt{\texcommand{usepackage}\{bookman\}} to see if that helps.
\item Use \unixcommand{pdflatex} to generate the PDF-file
  directly.  Included graphics may have to be converted to
  \texttt{png} instead of \texttt{eps}.  The PDF file generated from
  pdflatex is at least as good as the one generated with
  \texttt{dvips} and \texttt{ps2pdf}
%  The traditional approach with
%   \unixcommand{dvips} and \unixcommand{pstopdf} \textsf{from
% ;    Ghostscript}
\item Develop the document to work with both \unixcommand{latex} and
  \unixcommand{pdflatex}.  While writing use the traditional
  edit-compile-view cycle with \unixcommand{latex} and
  \unixcommand{xdvi} since it is much faster than with
  \unixcommand{pdflatex} and \unixcommand{acroread} (especially as
  acroread cannot reread a file without closing that window and
  opening the file again)
\end{itemize}

I have had problems with printing the PostScript file from
\unixcommand{dvips} directly to some HP-printers.  Results have been
much better with printing the PDF-file with
\myurl{http://www.adobe.com/products/acrobat/readstep.html}{Acrobat
  Reader} (can be done directly from the command-line with
\unixcommand{cat source.pdf | acroread -toPostScript | lpr} -- use the
\texttt{-helpall} option to get a full overview).


%\subsection{Including screen dumps}
%\label{sec:report-writing-tools-latex-eps}

Screen dumps have been captured with
\myurl{http://www.freshmeat.net/appindex/1998/07/28/901622941.html}{xv}
(which is now 6 years old and still the best general image tool on
Unix) and saved as \texttt{tiff} files.  These were then converted to
\texttt{png} with \unixcommand{pnmtopng} and to \texttt{eps} with the
\texttt{pnmtops} program in the 
\myurl{http://www.freshmeat.net/appindex/1999/06/24/930238715.html}{NetPBM
  package}, which has worked flawlessly.  It is important to remember
to use \texttt{pnmtops -noturn -rle} to get the correct orientation and
a reasonable compression (25 files shrank from 52 Mb to 9 Mb total).

The \texcommand{includegraphic} (from the \texttt{graphicx} package)
can read \texttt{png}-files from within \texttt{pdflatex}, and
\texttt{eps}-files from within \texttt{latex}


The process of embedding Postscript is fully described in The {\LaTeX}
Graphics
Companion~\cite{goosens-rahtz-mittelbach:the-latex-graphics-companion}.

% \begin{verbatim}
% [ravn@personal-53 gr]$ cat *.eps | wc
%  872378  873482 53187873
% [ravn@personal-53 gr]$ rm *.eps
% [ravn@personal-53 gr]$ make -s
% [ravn@personal-53 gr]$ cat *.eps | wc
%  149509  151878 9063019
% [ravn@personal-53 gr]$
% \end{verbatim} %$

I have had ``\texttt{xdvi}'' active while writing, and if the command
``\texttt{latex mydocument \&\& killall -USR1 xdvi}'' is used to
format the document, xdvi will automatically update after a successful
compilation without any further human interaction.  That is very nice
when your editing have happened within a single physical page.

For presentation purposes, the Danish TeX users group has a
\myurl{http://sunsite.auc.dk/dk-tug/presentation.html}{description
  regarding how {\LaTeX} can generate PDF-files} well suited for
presentations.

\subsection{Finding back to the source document from xdvi}

When reading the typeset document (either in \unixcommand{xdvi},
\unixcommand{acroread} or on paper) it is often a problem to find the
appropriate \texttt{tex}-file.  I used this trick which was deduced
from the \texttt{fancyhdr} documentation, which puts the
filename to the left in the running header on each page.

\begin{verbatim}
\usepackage{fancyhdr}           % provide headers where I can put the filenames.
% Trick from page 16 in fancyhdr documentation
\newcommand{\currentinputfile}{}
\newcommand{\myinclude}[1]{%
  \renewcommand{\currentinputfile}{File:\texttt{#1}}\include{#1}%
  \renewcommand{\currentinputfile}{}}

\lhead{\currentinputfile}
\pagestyle{fancy}

...
\begin{document}
\myinclude{s-front-matter}
\myinclude{s-introduction}
...
\end{verbatim}

There exists patches to xdvi which allow a user to click somewhere on
the xdvi window, which then tells Emacs to go to the specific place in
the original source corresponding to the clicked position.  My intial
testing proved it to be unstable for me, so I decided not to use it.
Your mileage may vary.


%\textsf{Modified xdvik}!

\section{Emacs} 
\label{sec:report-writing-tools-emacs} 


Emacs is a very well known editor in the Unix-world where its
extensability allows it to get new functionality by installing
additional modules.  The standard distribution of Emacs provides:

\begin{description}
\item[Ispell] -- The M-Tab command (bound to
  \texttt{TeX-complete-symbol}) runs \texttt{ispell-complete-word} if
  the current word is not a {\TeX}-command, which suggest possible
  words starting with the word under point.  The M-\$ command (bound
  to \texttt{ispell-word}) is a great help when in doubt about a
  word.  The \texttt{flyspell-mode} highlights words while you type,
  akin to the wavy underline used in MS-Word. 

\item[CVS] -- Emacs supports CVS directly.  I
  had my working directory on my portable, and my CVS repository on
  the MIP system, in order for my CVS files to be automatically backed
  up.  For this to work with ssh the following environment variables
  needs to be set
\begin{verbatim}
export CVSROOT=:ext:ravn@silkeborg.mip.sdu.dk:/home/ravn/CVS/SPECIALE
export CVS_RSH=ssh
\end{verbatim}
Use the \texttt{eval `ssh-agent`} and \texttt{ssh-add} commands before
starting X will provide full ssh-access for all programs started in X,
including Emacs thus providing full CVS support within Emacs.

\item[Compile] -- The \texttt{M-x compile} works well with Jade,
  nsgml, Java and other tools where you may encounter errors in your
  source.  \texttt{Ctrl-x `} makes Emacs go to the line with the
  error.
\item[Grep] -- The \texttt{M-x grep} and \texttt{M-x grep-find}
  commands are excellent for locating strings in your documents.
\end{description}

Additionally I have installed:


\begin{description}
\item[AUC-TeX] -- provides a complete IDE for writing {\LaTeX}
  documents, including keyboard shortcuts, a debugger, and intelligent
  commands depending on the state of the files. 
  
\item[psgmls] -- provides easy entering of tags, and parses DTD's to
  verify tags, and describe which tags are open as well as valid at a
  given cursor location.  Good for writing XML and SGML directly.  See
  \myurl{http://www.lysator.liu.se/projects/about_psgml.html}{the
    psgml homepage} for details.  I have considered writing a set of
  AUC-TeX compatible keybindings for DocBook in SGML-mode.
\end{description}


%\subsection{Making them all work together}
%\label{sec:making-them-all-work-together}

Using these packages has basically been very easy, but occasionally
they step on each others toes.  That could normally be circumvented:

\begin{description}
\item[{\TeX} and RCS] -- The RCS header fields, contain dollar-signs
  which causes {\TeX} to enter/leave math-mode (like
  \mbox{\$I}\mbox{d\$}), causing it to be rendered in math italic
  $which is not suitable for prose$.  By typing it as \texttt{\$
    \${}Id\$ \$} it is possible to avoid this. The dollar-space-dollar
  renders the space in math mode and switches back to normal text.
\end{description}

%%% Local Variables: 
%%% mode: latex
%%% TeX-master: "rapport"
%%% End: 

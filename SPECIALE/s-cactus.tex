% $Id$

%\chapter{Sample implementation -- Cactus}

% \mycitation{\txextsf{if purchasers [of word95] need to read [word97]
%   files they can buy an upgrade }.}{William H. Gates III}{\textsf{??}}


\chapter{Cactus - a web based publication conversion system}

\label{sec:cactus}
Today there are very few working procedures which are potentially
available to any computer user at a very low cost:

\begin{itemize}
\item Printing a document
\item Email an attachment to another user
\item Fax a document
\item Scan an image
\item Receive a voice mail
\end{itemize}

All of the above can easily be done with a computer as the recipient,
allowing it to capture the data sent.  The format preserving the most
information is email with an attached file, as the others use a visual
or audible representation of the original data.

\section{Overview}

   My idea with Cactus was from the start to create a web-based facility
   which would ease the process of publishing and sharing information via
   the Internet.

   [The short comings of Yggdrasil]

   I set the following goals for Cactus:
     * Publishing documents must be as easy as possible for authors
     * The publishing process should be fully automatic, without the need
       for human web masters
     * The software requirements of the audience should be as small as
       possible
     * The system must run under Linux

   These goals have been fullfilled to my personal satisfaction even
   though the program turned out differently than I originally
   envisioned. Cactus is today a system which
     * accepts electronic documents from many sources
     * applies filters repeatedly to convert each document to as many
       other forms as possible
     * allows easy access to the documents through a webbrowser
     * provides a public service - all information is available to
       everyone
     * allows documents to expire if they are not intended for long term
       storage


   Philosophy




   Cactus seen from the users perspective


   Design

   Overall view

   Data flow

   Break down in tables and processes

   Good decisions. Bad decisions.


   Implementation

   Considerations language, SQL database, platform,


   Source code

   If I get the time I will make this a literate form. Perl modules with
   the literate programing extension?







\chapter{Go away}




% \subsection{Yggdrasil - a simple navigational framework}
% \label{sec:yggdrasil}


% The Yggdrasil system was designed to provide a simple, consistent
% navigational framework around a dynamic set of web pages provided by
% users, as well as providing a "recently changed pages"-list plus an
% overview of pages written by each person.  This was a very successful
% idea in the start, but gradually lost momentum due to these factors:

% The development platform was changed from Unix to NT, which made it
% much more difficult for the users to access their web-directories, as
% these were no more a part of their home directory\footnote{The Apache
%   webserver uses the ~/public\_html directory for the users personal
%   web pages.  The transition to NT meant that the users - in addition
%   to their normal file manipulations - should telnet to a Unix server,
%   and change the file attributes every time the file was updated.  }

% The internal document format became Microsoft Word, which at that time
% could not be converted to HTML.  Such documents were therefore unable
% to be published.

% The users could trigger an update by sending an empty email to the
% system, as well as rely on an automatic nightly update.  The trigger
% mechanism was not brought along when the email system was converted to
% NT.

% New users was not informed about the system.














ypx is from comp.sources.misc volume 40
http://ftp.lth.se/archive/usenet/comp.sources.misc/volume40/ypx/

compiles on solaris with "-lsocket -lnls".  does not compile on linux



\section{Background}

A major computer problem not solved satisfactory yet, is the ability
for users to share information in spite of them using different
hardware and software.   Even though the modern use of the Internet
allow any two users to exchange files as attachments to
email
without any furter ado, it has not helped much in actually \textit{interpreting} the
contents of these files.

If a given user needs to use a given document, she needs software
specific to the document format to interpret it.  If another user
needs the document, she needs software too.

The most prominent word processor document format today is the
Microsoft Word format, which is so complicated that even Microsoft has
trouble with it (hence the quotation of this chapter).  Basically you
need Word to use this format -- all other solutions provide inferiour
results.  If you only need to read and print documents Microsoft
provides a Word Viewer program for this, which require Microsoft
Windows.

Unfortunately, this is no help for those users who does not have Word,
notably Linux users.

\textsf{presentation and internet thingie}




The Cactus system is an Open Source document conversion and
presentation framework, which allows users to submit documents to a
central server, which then uses a set of stored conversion filters to
process the documents into other forms, notably HTML and other
Internet formats.

The version of Cactus described herein, have the following features:

\begin{itemize}
\item Convert documents to HTML versions viewable in a browser
\item Windows and Unix clients can use Cactus as a virtual PostScript
printer with automatic PDF conversion, and may download the result
directly or view the HTML-conversion (making this accessible to
platforms without Acrobat Reader).
  
\item Images can be resized and converted to a number of formats like
  JPEG, TIFF, and PNG.
\item Extensible and configurabele.  New filters can be installed to
  convert from one MIME-type to another. 
\end{itemize}


This runs on a single Linux machine, but any number of Unix machines
may requests jobs to do if more CPU-power is needed.   \textsf{The
  necessary filters and a simple Java client must be installed on
  each.  Will the support be active?}.





\textsf{ingen dokumentstandard.  ingen wp-producenter der underst'tter
  det, ingen interessei at goere noget ved det, ingen ting.

  nu wp9 understoetter sgmlredigering, StarOffice er frit
  tilgaengeligt paa mange platforme, osv osv}



% \section{abstract}

% \begin{quotation}
%   The Cactus system fills a gap in the current integration of the web
%   with normal office procedures, as it provides easy web publishing
%   for occasional authors, by letting them submit documents in several
%   ways with their usual software.

%   The system automatically produces other versions of the documents on
%   demand of the users reading the documents, as well as provide the
%   navigational framework, relieving the local webmaster from these
%   tedious and error prone tasks.  
% \end{quotation}

\subsection{System description}
squid:  http://www.squid-cache.org/

\framepage{15cm}{
Cactus is a sample implementation of the following issues:

\begin{itemize}
\item \textbf{easy publishing} - users can publish via
email/fax/print/watch usenet/www which is processed into the SQL
database, and confirmed.

\item \textbf{automatic conversion} - system will automatically
convert a given document to what the user can see (or want).  Browser
sends a capability string with each request - use that to provide
stuff directly, or generate a ``this is available'' summary.

\item \textbf{automated navigational framework} - each document has an
  annotation which tells Cactus where to place it in the navigational
  hierachy.  The corresponding navigational pages are automatically
  generated and updated when new documents arrive.  This also ensures
  system integrity without ``broken links''.
\end{itemize}

Implementation languages - possibly Java or Perl.  Perl chosen due to
better library support (with source).  Rex thingie in Java.  How can
MIME, TAR, etc be done in Perl (remember jubilations!).  Using Apache
with Cocoon (perhaps) and MySQL.  Graph algoritms
in~\cite{sedgewick-algorithms-in-c}.

Good summary from 19991114.    Speed of PNG gzipping?  PNG
uncompressed intiially and then later compressed by pngcrunch.
}

http://photo.net/wtr/word.html
Utilities:  mswordview, xls2xml

\subsection{Background}

\framepage{15cm}{ Explain the history.  Yggdrasil to address the need
  of an automatic webmaster $\rightarrow$ analysis (seperate file
  written earlier).  Cactus to address the difficulties of publishing
  information to Yggdrasil.}


Explain Cactus as a successor to Yggdrasil.

\subsection{Installation}
Perlmodules.  MySQL server. SAMBA.


[Rephrase following paragraph]

I have concluded the following goals for CACTUS, in order to avoid
repeating history:

\begin{enumerate}
\item "Ease of publishing" is crucial.
  
\item Document preparations and transformations must be fully
  automatic.
  
\item The organisation using the system must be fully doing so.

\end{enumerate}

\subsubsection{"Ease of publishing" is crucial}

CACTUS uses two approaches in order to make publishing as easy as
possible for the users, namely

\begin{center}
  Documents are accepted in their native format, and in numerous ways.
\end{center}

Discussions with potential users showed [...] to be realistic ways in
which CACTUS would be used:

As a document storage for various versions of a document, being
developed and emailed back and forth between authors and peers.  By
allowing CACTUS to accept documents as email attachments, it would be
very easy to enter each draft in CACTUS by including it on the list of
authors or peers.  In this way the archival of the document in CACTUS
is completely transparent.  As a fax machine.  When replacing a fax
machine with a computer, it is very easy to send a copy of the
temporary image of the fax to CACTUS, before printing it.  The fax
system is then enhanced with the possibilities of the web, relieving
the need for the physical copy.  As a printer.  Cactus provides a
"printer", which makes a web-version out of any document the users can
print.  Users can then share final versions of documents without
requiring the recipients to have the software in question.  Either the
Adobe Acrobat Reader can be used, or the primitive multiple image
viewer in Cactus.

The full list of the publishing methods in Cactus is listed [....]

The users were primarily expecting to use these file formats:

\begin{itemize}
  
\item 
  Word DOC and RTF files.   These are the storage formats of the
  Microsoft word processor Word.  
\item 
  HTML files.  The common format for the web.
\item 
  LaTeX files.   This typesetting program is very popular with
  mathematically oriented academics.
\item 
  GIF, JPEG, TIFF and PNG images.   These and many more are in common use.  Cactus use PNG internally [except possibly for JPEG].  The full list of supported
formats is listed in the implementation section [see somewhere].
\item 
  PostScript and PDF files.
\item 
  Fax images.
\end{itemize}

\subsubsection{Document preparation and conversion must be fully automatic}

There was a strong consensus amongst the users, that it would be very
nice not to have to convert the documents manually every time they
were to publish a document.  Therefore Cactus accepts several document
formats as described above, and abandons the requirement that the
users should convert their documents to HTML before publishing.

The system have some very different views on the documents depending
on which representation the user needs - not all make sense for all
documents:

\begin{enumerate}
\item The unaltered original.  This file is always available, allowing
  users with the correct software to continue working with it.

\item A normalised version of a document.  This could be a PNG version
  of a TIFF image or BMP image which can be viewed by all modern
  browsers, and an XML version of a document.  If a suitable encoding
  can be found, even images and sound can be represented in XML so
  that this does not overlap the textual representation.

\item A visual representation of the original.  This is the "look of
  the file", i.e. as it would look when printed to paper, and allows
  users without the corresponding software to view and print the
  contents.

\item A textual representation of the original.  This is the "meaning
  of the file", which could say that the line "foo bar" is a level 2
  heading, providing search capability.

  
\item An audio representation of a "document".  This is e.g. an
  message left on an answering machine.
\end{enumerate}

Conversions between all these document representation must be
automated as much as possible.  [write about the Cactus framework for
providing existing and future filter types].  The normalised document
is the only one created when a document enters Cactus - the rest are
created on demand to avoid overfilling the underlying database, but
cached for a reasonable time to improve performance.

[....]


The organisation using the system must be doing fully so.[ rephrase]

In order for such a system to be successfully used within an
organisation, it is vital that the organisation is using it
whole-heartedly.  [and more of the same].


Note on derivers:

A deriver is a pipeline which derives another version of an item.
E.g. PNG to GIF,

There are the following tasks:
\begin{description}
  
\item[\textsf{Validators}] -- validates whether the content of the
  item is conforming to the MIME-type provided and the data valid.  If
  not, a MIME-type is synthesized according to filename and content.

%   If
%   not, or if a MIME-type is not provided, an guess is made to the
%   MIME-type based on the filename and file content, and this
%   synthesized MIME-type is then validated.  If that fails 
  

% conforms with the is consistent with the MIME-type, and is it conforming to the
%   standard.  (Can a gzip-stream be decompressed? is a PostScript file
%   parsing correctly? etc.).  If a type is unknown, guess MIME-type
%   from filename and/or contents, and validate it.  If all fails,
%   assign a type of application/octet-stream.

\item[Extractor] -- looks \textit{inside} an item to look for
references, embedded files and other kinds of extractable data.  A
reference may be an URL or an emailaddress in a signature.  Embedded
data could be an uuencoded image in a Usenet posting.

\item[Converters] -- create another version of a given item fast.  These
  are intended for conversions of data with users waiting, like
  \texttt{pnm} to \textit{uncompressed} \texttt{png}.  The generated
  versions are stored in the database and compressed when the
  system is otherwise idle. 

%   These should generally be designed
%   to be as fast as possible, since these will be called from
%   CGI-programs with users waiting.  Avoid compression (gzip should
%   maximally be level 1).

\item[Derivers] -- create another item from one or more originals
irreversibly, like generate \texttt{dvi}-files from several source
files.  A deriver can ask for a conversion.  For efficiency reasons,
several computers may run derivers for Cactus.

%   This could be a multi-part MIME file which should be assembled into
%   a fresh original.  It could be a DVI file from several source files
%   (tex files and images).  These should generally be reasonably fast,
%   since their speed specifies how fast a new item can be made
%   available to the system.

\item[Compressors] -- Reduce the size of a converted item if possible,
by reencoding the data using any built-in compression schemes in the
data format of the item.  Sample formats are \texttt{png} and
\texttt{tiff}.
  
% A converter is usually designed for being usable in
%   a real-time interactive setting (also called being fast), which
%   normally means that the result is sub-optimal.  An optimiser
%   complements this by doing a suitable optimisation step whenever the
%   system is sufficiently idle.  This could be running "pngcrunch" on
%   PNG files (which yields 30\% on Ghostscript output), or "tifftotiff"
%   on TIFF files.  The idea is that the file is still the same basic
%   type, and can be used without further modification.  It is an
%   "in-place optimisation".  Some file formats are using the gzip
%   compression scheme internally.  These would not benefit further from
%   archiving.
  
\item[Archivers] -- moves compressed items which has not been used for
a long time into long-term storage, leaving only the metadata.  This
keeps the working data set small.

%   This would typically be running "gzip -9" on the content, creating a
%   new entry (with a new mimetype), and marking the item as decachable.
%   The dearchiving process must be fast, since it might be needed by a
%   deriver without notice.
  

\end{description}

These are complemented by

\begin{description}
\item[Acquirers] -- gathers items from "outside" Cactus and enters it
in the "incoming" table with an appropriate MIME-type, an expiration
date, and \textsf{misc user information}.  An acquirer could accept
email, emulate a printer, retrieve Usenet articles, etc.

\item[Janitors] - cleans up whenever the system is otherwise idle, or
  when the cache is full.  Converted items can be emptied of content,
  originals can be archived.  Expired items can be purged completely
  from the database, along with all their derived items.

\item[Presenters] - extracts data, and present them.  This could be a
  CGI-script presenting a given item as a http-stream.  A PDF item
  could be presented as a window with two frames, the leftmost
  containing a thumbnail pr page, and the rightmost a high resolution
  version of a given page.  The page should be cut in smaller pieces
  to allow easier processing by Unix Netscape.
\end{description}

In an ideal world we have infinite storage and infinite CPU-speed,
meaning that everything would be done instantly when we need it.  In
order to decide the order in which to do tasks, a price system must be
developed which would guarantee that the system processes every item,
that the system is still responsive while converting,

\textsf{
What to do when the system runs so full that archiving cannot be done
fast enough.  Can data be moved to "outside storage?".
}


\subsection{The implementation of Cactus.}



Goals:

\begin{itemize}
\item Stable, well-known and remotely administrative platform-
  Linux/Solaris
\item Platform independence - the resulting system must not be tied to
  Unix
 
\item Modular - in order to minimise actual development, software
  libraries should be used as much as possible.
  
\item Extensible - it should be easy for the system administrator to
  enhance functionality.
\end{itemize}

Choosing an implementation language:
\label{sec:cactus-choice-of-language}

Since platform independence was important for the system, it was
quickly found that reasonable choices would include scripting
languages plus Java.  Scripting languages are fully interpreted
allowing a script to run on numerous platforms without change.  Java
is the only compiled language with this property, due to the "Write
once, Run everywhere" philosophy from Sun, plus the tight integration
with Internet technologies in Java.

Initially I wanted to write the core of Cactus in Java, and spent a
couple of weeks evaluating the language but found that


  The mentality of the Java-community on the Internet, is very influenced by the shareware philosophy typical for the PC-user.   Everything useful cost money, source
code is not revealed, and the general tendency is for large, stand-alone applications.

Nobody had written a publicly available, stand-alone MIME-parser in Java.  Several RFC's should be implemented and tested, in order to get email-parsing in Cactus.

The general level of abstraction is - in my opinion - too low in Java, while the "core language" is enormous.



[?]



Information extraction from items.



A given document contains one or more items, which again may contain
further information.  Cactus uses the MIME-type of a given item, to
select the information scanning method with a fall back to the generic
application/octet-stream examiner.

The application/octet-stream is parsed for:

The text is scanned for URL?s, either with the Tom Christensen urlify
or the program posted or commented by Abigail.  These include ftp,
http, gopher, mailto and news references.  These are stored as
external references.

The text/plain is parsed for

uuencoded data in a text stream.  These start with ?begin \#\#\#
name?, contain lines matching \#\#\#\#, and ends with ?end?.  Such a
file is then assigned a validated MIME-type based on the name of the
file, and entered in the system as a derived item.

An text/html item is parsed for:

normal references in \tag{a}-anchors.  While doing this, the text to
be rendered is extraced and parsed as a text/plain stream, in order to
get the sequence right (may be changed).  The \tag{meta} tags are
examined in order to extract keywords for the label.



\subsection{Converters}
\label{sec:converters}


\subsubsection{Microsoft Word}

The Microsoft Word DOC-format is widely used, but apparently so hard
to use that even Microsoft have trouble doing it correctly, and the
reason why Cactus was started in the first place.  I have spent a
great deal of time looking for suitable software which could have been
used with Cactus on the server side, and then testing it out.
\textsf{WINE}

\begin{description}
\item[Microsoft Word] -- The best solution would be able to actually
  run Word itself, but apparently this is integrated so well with
  Windows that it cannot run on other systems like
  \myurl{http://www.winehq.org}{WINE}.  Since Microsoft previously
  have deliberately made the Windows 3.1 version of Internet Explorer
  4 unable to run in the WinOS/2 subsystem for OS/2, I strongly
  suspect that this is also the case here.
  
\item[Viewer for Microsoft Word] -- the Word Viewer is available for
  all Microsoft operating systems.  The 16 bit version is reported by
  Usenet posters to behave reasonable in WINE, and would be a good
  candidate for producing PostScript printouts of Word documents.  It
  would require a full Windows application to automate this, since it
  does not support command line arguments.

\item[Corel WordPerfect 8 for Linux] - This is generally a very nice
  word processor, but the import filter for Word crashed WP when I
  tried it with a large document.  Filters should be better in
  WordPerfect 9, which is due for Linux medio 2000.
  
\item[StarOffice 5.1] -- the German office suite for several platforms
  have excellent filters for importing Office files in general, but
  cannot be automated from the commandline.  A StarBasic program must
  be written and invoked to do the job -- a request on the StarOffice
  newsgroups for a tool to do this, did not get any response.  Due to
  lack of time I did not pursue this further.
  
  Since I did the testing, Sun have bought StarOffice - most likely
  since they need an network based office suite for Java, which
  StarOffice provide with thin clients and a Solaris based server -
  and promise to make the source generally available.  This has not
  happened yet, but it is currently the most likely alternative to the
  Microsoft Office solution.
  
  The StarOffice package is such a popular package that Sun has a
  download counter on their main home page (which said 1,962,277
  downloads as of 2000-04-03).
  
  
\item[AbiWord] -- An Open Source word processor which is part of the
  Gnome Office suite.  It was not able to parse my test documents in
  Word.
  
\item[wv] (previously mswordview) -- an Open Source Word to HTML
  converter which currently do text well, but have trouble with graphs
  which are converted to the WMF format.

\item[Do-It-Yourself] - If you ask Microsoft very nicely, you can get a
  copy of the Microsoft Office Binary File Format Specification, and
  write a personal parser of Word.  It took Microsoft 6 months to
  answer my email request, and then they just sent me the license
  twice with no specification.  Second time they got it right.
  Unfortunately, the license was so restrictive that I decided not
  even to \textit{look} at the specification.
  
\item[Majix] - This small RTF to XML converter written in Java, can
  also parse DOC files when running in the Microsoft Java Machine by
  invoking Word to save the DOC file as RTF.  (Unfortunately it had
  serious problems with RTF files generated by other programs).  It
  was easy to customize it to generate DocBook XML.
  
  If Majix could be called from Cactus with a DOC-file and return the
  corresponding XML file a major goal had been accomplished.  A test
  scenario was therefore made with a remote telnet session to a
  NT-machine, where Majix was invoked from the command line.  The test
  failed as Word hung in the start up phase -- my personal guess is
  that Word needs access to the GUI.  An installation on a MIP server
  was not possible due to MIP system policy.
  
  Additionally the company has not released a new version in a year,
  and the license explicitely prohibits disassembly. 

\end{description}

My intermediate solution has been to adapt the mswordview to output
DocBook XML instead of HTML, which provides a text-only
display.


\subsubsection{{\TeX} and {\LaTeX}}

\textsf{tex4h}


\subsection{MySQL}
On asserballe

\begin{verbatim}


create table incoming (when datetime, mime varchar(80), user
  varchar(80), how varchar(80), filename varchar(255), comment varchar(80),
  item longblob);

+----------+--------------+------+-----+---------+-------+
| Field    | Type         | Null | Key | Default | Extra |
+----------+--------------+------+-----+---------+-------+
| when     | datetime     | YES  |     | NULL    |       |
| mime     | varchar(80)  | YES  |     | NULL    |       |
| user     | varchar(80)  | YES  |     | NULL    |       |
| how      | varchar(80)  | YES  |     | NULL    |       |
| filename | varchar(255) | YES  |     | NULL    |       |
| comment  | varchar(80)  | YES  |     | NULL    |       |
| item     | longblob     | YES  |     | NULL    |       |
+----------+--------------+------+-----+---------+-------+
7 rows in set (0.01 sec)


\end{verbatim}

\begin{itemize}
\item \textbf{ wmf-anything}: libwmf (no fonts),
  
  \myurl{http://ourworld.compuserve.com/homepages/kkuhl/}{KVEC} (could
  not render the wmf files from wv),
  

  \myurl{http://www.companionsoftware.com/PR/WMRC/WindowsMetafileFaq.html}{WMF
    docs}
  
\end{itemize}


\section{Programs}


\subsection{Linux - an operating system}
\label{sec:linux}

Linux was chosen as the development platform for these reasons:

\begin{itemize}
\item High degree of familiarity with the operating system
\item Almost all Open Source software run ``out-of-the-box'' on Linux
\item High performance Java Development Kit available from IBM
\item Remotely maintainable via X/telnet/ssh.
\item Freely available from the Internet.
\item Root access possible without interfering with MIP security rules
\end{itemize}

These rules were also most likely fullfilled by any of
FreeBSD/NetBSD/OpenBSD/Solaris x86, but I had already burned Redhat
6.1 on a CD for my portable, so there was no need to look any
further.  Linux has been a very satisfactory choice.

\subsection{Apache - a high performance web server}
\label{sec:apache}

There have probably been written more webservers than editors in the
world of today.  My requirements were simple:

\begin{itemize}
\item Run Perl programs efficiently
\item Run Java servlets efficiently
\item Have a large user base in order to ensure program availability
\end{itemize}

This basically left a single web-server, namely
\myurl{http://www.apache.org}{Apache}, which is capable of running
Perl CGI scripts very efficient with
\myurl{http://perl.apache.org/}{the \texttt{mod\_perl} module} (which
uses a resident Perl interpreter combined with caching of the bytecode
of previously encountered programs).

If requirements were less, other webservers might have been
applicable.  The
\myurl{http://www.mortbay.com/software/Jetty.html}{Jetty} (Java
Webserver capable of running servlets),
\myurl{http://www.acme.com/java/software/Acme.Serve.Serve.html}{Acme.Server}
and \myurl{http://www.roxen.com}{Roxen} (standard webserver with
excellent graphics capabilities) webservers are just a few candidates for
tasks with a less diverse need for scripting languages.  Servlet
support is getting very common place.


\subsection{Squid - a high performance web cache}
\label{sec:squid}

\myurl{http://www.squid-cache.org/Doc/FAQ/FAQ-20.html\#what-is-httpd-accelerator}{The
  http-accellerator mode}


\subsection{Cocoon}
\label{sec:cocoon}



\section{Filters}
\label{sec:cactus-filters}


\subsection{ps to pdf}
\label{sec:filter-ps-to-pdf}

\begin{description}
\item[\myurl{http://www.adobe.com/products/acrdis/main.html}{Adobe
    Distiller}] -- The reference PostScript to PDF converter.
  
\item[\myurl{http://www.ghostscript.com/}{Ghostscript}] -- This is a
  PostScript interpreter which has gained wide usage.
\end{description}

\textsf{??} rtf2latex http://www.tex.ac.uk/tex-archive/support/rtf2latex2e/


papirbjergene ImageTag, 3m xerox

\section{Scanning in mail}

\textsf{ELSAM}




%%% Local Variables: 
%%% mode: latex
%%% TeX-master: "rapport"
%%% End: 

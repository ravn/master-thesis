% $Id$

\chapter{Terms and concepts}
%\section{Terms and concepts}
\label{cha:terms-and-concepts}

\mycitation{HTML is a SGML DTD!!}{Me}{1994}


This report uses a lot of abbreviations, many of which may not be
widely known.  This section lists most of them.


\textsf{The list will be sorted -- steal XML graph from LteX Web
  companinon.  Check that all abbrevs are expanded}

\begin{description}

\item[entity] A named ``unit'' in XML/SGML which expands to a string.
  This is very similar to a macro without arguments in other
  languages.
  
\item[XSL-FO] XSL-formatting objects.  A generic description of the physical
  layout of a XML-document on paper.  These files can be converted to
  PDF with FOP or PassiveTeX
  
\item[SGML] A family of languages.  The \textit{DTD} specifies which
  language it is.  SGML can be formatted with FOSI or DSSSL
  style-sheets.  \unixcommand{jade} can format to HTML, plain text and
  RTF. 
  
\item[XML]
% A family of languages.  The \textit{DTD} specifies which
%   language it is.  XML-documents can be transformed with
%   \textit{XSL}-style sheets to another XML-document (this process is
%   called \textit{XSLT}).  XML is a subset of \textit{SGML}
  
XML is a light-weight version of SGML (both defines document
languages) designed to be embeddable in a browser.  XML-documents
\textit{may} comply to a DTD, or be stand-alone (``DTD-less'').  XSL
is used to convert a XML document into another XML document (XSLT) or
into the generic page description language XSL-Formatting Objects
(XSLFO).


\item[*ML] Common description of both XML and SGML, where the two have
  the same applications.

\item[DTD] The Document Type Description is the specification
  which describes the exact syntax of an *ML document.

  
\item[XSLT] Either the process of \textit{transforming} a XML-document
  into another XML-document using a XSL-stylesheet, or the process of
  \textit{formatting} a document into FO, which then can be further
  formatted in PDF.
  
\item[DOM] The Document Object Model is W3C's first model for
  representing an XML tree internally.  It requires the whole XML
  dataset to be read into memory which is a limitation for large
  projects.  The later SAX interface is more lenient in its memory
  requirements.
  
\item[SAX] Simple API for XML.  An event based approach to XML-parsing
  and representation.  Has an advantage in that the design allows
  processing to begin before the whole XML-tree has been read in.
  Parsers and processors which implement SAX can be interchanged
  freely, currently allowing for 20 different combinations of parsers
  and processors.


\item[HTML] Hyper Text Markup Language.  The ``language of the web''
  which was originally designed by a physicist to present articles to
  other physicists.  Was later hacked upon by Netscape and Microsoft
  to do things it was never originally meant to do.

  
\item[DSSSL] SGML Document Style Semantics and Specification Language.
  It is used to render an SGML document to another format.
  \unixcommand{jade} can render to RTF, {\TeX} (must be
  post-processed with \unixcommand{jadetex}), SGML and XML.
  
\item[FOSI] \textit{Formatting Output Specification Instance} --
  Another style sheet for SGML, which I do not know anything about.
  \myurl{http://navysgml.dt.navy.mil/dtdfosi/repository.html}{The US
    Navy have a big repository}.

  
\item[SQL] The Structured Query Language is the standard
  language for communicating with a database.

  
\item[XSP] \texttt{XML Servlet Pages} - A technique for combining code
  with XML in a single page, which is parsed and executed by the
  webserver when it is requested.
  
\item[RTF] \textit{Rich Text Format} - a document description language
  designed by Microsoft.  Widely supported.  May contain extensions to
  the original RTF-specification.

\item[Jade] A DSSSL-engine for SGML documents by James Clark.
  
\item[JPEG] \textsf{??} -- An efficient method for representing photographs in
  computer files.  Efficiency is achieved by discarding the least
  visible parts of the visual information to the human eye.
  
\item[GIF] \textsf{??} -- An image format well suited for computer
  generated images with few distinct colors, like icons.  Is hampered
  by a patent on the compression scheme used and a maximum of 256
  colors.  Superceeded by PNG.
  
\item[PNG] \textsf{??} -- The successor to GIF.  Is supported in newer
  browsers only.
  
\item[CGI] The Common Gateway Interface.  The original way to generate
  pages and other files dynamically.  A CGI-script can be written in
  any language supported by the web server.

\item[IIS] Microsoft Internet Information Server.  The webserver from
  Microsoft. 
  
\item[ftp] File Transfer Protocol.  This is both the name of the
  \textit{protocol} (the method) as well as the \textit{program}
  implementing the protocol.  ftp can transfer files between a client
  computer and a server computer, and is a standard on the Internet
  
\item[http] Hyper Text Transfer Protocol.  This is the protocol a web
  browser needs to speak with a web server, in order to retrieve
  documents.  Http is intentionally very simple.
\item[TCP/IP] This is the procotol which a computer needs to speak to
  connect to the Internet.  It allows two computer to establish a data
  stream, where one machine pushes bytes in one end of the stream, and
  the other machine retrieves the bytes from the other end of the
  stream, without either computer being concerned about the underlying
  network.
  
\item[SP] \myurl{http://www.jclark.com/sp/index.htm}{SP is a SGML parser} by James Clark.
  
\item[Perl] A scripting language which has become popular with Unix
  system administrators, and which ``was there'' when a need for
  scripting languages for CGI arose.  Has eminent regular expressions
  and is generally well-suited for handling text.
  
\item[ASP] \textsf{??} -- Microsofts version of a scripting language
  intertwined with HTML. Works on Microsoft web servers only.
  
\item[JSP] Java Server Pages.  Suns version of Java intertwined with
  HTML.  Requires a servlet-capable web server.
  
\item[PHP] \textsf{??} -- The Internet version of a scripting language
  intertwined with HTML.  Can run as a CGI script in most browsers,
  but as a module (which is faster) in at least Apache.
\item[Apache] Open Source webserver which runs on a lot of different
  platforms.  Most Linux distributions include it.

\item[PassiveTeX] A package for {\TeX} by Sebastian Rahtz which allows
  FO files to be typeset by PDF{\TeX} directly into PDF.  PassiveTeX
  understands MathML, but does not yet support tables.
  
\item[MIME]
  \myurl{http://www.oac.uci.edu/indiv/ehood/MIME/MIME.html}{Multipurpose
    Internet Mail Extensions} which extends the format of Internet
  mail to allow foreign character sets and inclusion of files.  
  

\item[MathML] a XML-dialect to describe mathematical formulas.


\item[DocBook] A set of DTD's freely available from the Internet.
  See~\vref{sec:docbook}.
\item[TEI] A DTD freely available from the Internet.  
\item[MIP] The Maersk Mc-Kinney Moller Institute for Production Technology

\item[PostScript] -- a page description programming language, which
  was later evolved into EPS by constraining it to a single page, and
  PDF by constraining it not to contain code but only data.
  
\item[PDF] Portable Document Format -- a document description
  language by Adobe which builds on the experiences with PostScript.
\end{description}

  
    
%%% Local Variables: 
%%% mode: latex
%%% TeX-master: "rapport"
%%% End: 

% $Id$

\section{Terms and concepts}

\begin{quotation}
  ``HTML is a SGML DTD''
\end{quotation}

This report uses a lot of abbreviations, many of which may not be
widely known.  This section lists mosts of them.

\begin{description}

\item[entity] A named ``unit'' in XML/SGML which expands to a string.
  This is very similar to a macro without arguments in other
  languages.
  
\item[FO] Formatting objects.  A generic description of the physical
  layout of a XML-document on paper.  FOP converts this to PDF.
  \textsf{Several users have expressed doubt about this XSLT
    transformation}
  
\item[SGML] A family of languages.  The \textit{DTD} specifies which
  language it is.  SGML can be formatted with FOSI or DSSSL
  style-sheets.  \unixcommand{jade} can format to HTML, plain text and
  RTF. 
  
\item[XML] A family of languages.  The \textit{DTD} specifies which
  language it is.  XML-documents can be transformed with
  \textit{XSL}-style sheets to another XML-document (this process is
  called \textit{XSLT}).  XML is a subset of \textit{SGML}


\item[*ML] Common description of both XML and SGML, where the two have
  the same applications.

\item[DTD] The \textsf{Document T? Description} is the specification
  which describes the exact syntax of the *XML


\item[XSLT] Either the process of \textit{transforming} a XML-document into
  another XML-document using a XSL-stylesheet, or the process of
  \textit{formatting} a document into FO, which then can be further
  formatted in PDF.

\item[DOM] \textsf{W3C's initial model for representing an XML tree
    internally.  Superceeded by SAX (\textsf{for java?)}}
  
\item[SAX] Simple Application \textsf{interface for?} XML.  An event
  based approach to XML-parsing and representation.  Has an advantage
  in that the design allows processing to begin before the whole
  XML-tree has been read in.  (\textsf{URL?}).  Parsers and
  \textsf{processors?} which implement SAX can be selected freely,
  currently allowing for \textsf{20 different combinations of parsers
    and whashallicallem}.


\item[HTML] Hyper Text Markup Language.  The ``language of the web''
  which was originally designed by a physicist to present articles to
  other physicists.  Was later hacked upon by Netscape and Microsoft
  to do things it was never originally meant to do.

  
\item[DSSSL] \textsf{SGML style sheets something}.  Is used to render
  an SGML document to another format.  \unixcommand{jade} can render
  to \textsf{FOT}, RTF, {\TeX} (must be post-processed with
  \unixcommand{jadetex}), SGML and XML.  \textsf{Is this a standard?}
  
\item[FOSI]  \textsf{Another style sheet for SGML}, which I do
  not know anything about.  Apparently the US Navy uses it a lot.
  Check in DocBook.

  
\item[SQL] The Structured Query Language is the \textsf{standard}
  language for communicating with a database.

  
\item[XSP] \texttt{XML Servlet Pages} - A technique for combining code
  with XML in a single page, which is parsed and executed by the
  webserver when it is requested.

\end{description}

\framepage{15cm}{
Explain XML, XSL, XSLT, HTML (yes!), SQL, SAX, DOM, XSP and whatever
fancy terms come to mind.  Draw graph of what happens with an XML
document.}

    
%%% Local Variables: 
%%% mode: latex
%%% TeX-master: "rapport"
%%% End: 

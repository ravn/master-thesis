% $Id$

\chapter{Terms and concepts}
%\section{Terms and concepts}
\label{cha:terms-and-concepts}

\mycitation{``HTML is a SGML DTD''}{\textsf{?}}{\textsf{?}}


This report uses a lot of abbreviations, many of which may not be
widely known.  This section lists mosts of them.


\textsf{The list will be sorted}
\begin{description}

\item[entity] A named ``unit'' in XML/SGML which expands to a string.
  This is very similar to a macro without arguments in other
  languages.
  
\item[XSL-FO] XSL-formatting objects.  A generic description of the physical
  layout of a XML-document on paper.  These files can be converted to
  PDF with FOP or PassiveTeX
  
\item[SGML] A family of languages.  The \textit{DTD} specifies which
  language it is.  SGML can be formatted with FOSI or DSSSL
  style-sheets.  \unixcommand{jade} can format to HTML, plain text and
  RTF. 
  
\item[XML]
% A family of languages.  The \textit{DTD} specifies which
%   language it is.  XML-documents can be transformed with
%   \textit{XSL}-style sheets to another XML-document (this process is
%   called \textit{XSLT}).  XML is a subset of \textit{SGML}
  
XML is a light-weight version of SGML (both defines document
languages) designed to be embeddable in a browser.  XML-documents
\textit{may} comply to a DTD, or be stand-alone (``DTD-less'').  XSL
is used to convert a XML document into another XML document (XSLT) or
into the generic page description language XSL-Formatting Objects
(XSLFO).


\item[*ML] Common description of both XML and SGML, where the two have
  the same applications.

\item[DTD] The \textsf{Document T? Description} is the specification
  which describes the exact syntax of the *XML


\item[XSLT] Either the process of \textit{transforming} a XML-document into
  another XML-document using a XSL-stylesheet, or the process of
  \textit{formatting} a document into FO, which then can be further
  formatted in PDF.

\item[DOM] \textsf{W3C's initial model for representing an XML tree
    internally.  Superceeded by SAX (\textsf{for java?)}}
  
\item[SAX] Simple Application \textsf{interface for?} XML.  An event
  based approach to XML-parsing and representation.  Has an advantage
  in that the design allows processing to begin before the whole
  XML-tree has been read in.  (\textsf{URL?}).  Parsers and
  \textsf{processors?} which implement SAX can be selected freely,
  currently allowing for \textsf{20 different combinations of parsers
    and whashallicallem}.


\item[HTML] Hyper Text Markup Language.  The ``language of the web''
  which was originally designed by a physicist to present articles to
  other physicists.  Was later hacked upon by Netscape and Microsoft
  to do things it was never originally meant to do.

  
\item[DSSSL] \textsf{SGML style sheets something}.  Is used to render
  an SGML document to another format.  \unixcommand{jade} can render
  to \textsf{FOT}, RTF, {\TeX} (must be post-processed with
  \unixcommand{jadetex}), SGML and XML.  \textsf{Is this a standard?}
  
\item[FOSI]  \textsf{Another style sheet for SGML}, which I do
  not know anything about.  Apparently the US Navy uses it a lot.
\textsf{  Check in DocBook.}

  
\item[SQL] The Structured Query Language is the \textsf{standard}
  language for communicating with a database.

  
\item[XSP] \texttt{XML Servlet Pages} - A technique for combining code
  with XML in a single page, which is parsed and executed by the
  webserver when it is requested.

\item[RTF] \textit{Rich Text Format} - a document description language
\textsf{designed by microsoft}.  Widely supported.  May contain
extensions to the original RTF-specification.

\item[Jade] A DSSSL-engine for SGML documents by James Clark.

\item[JPEG] An efficient method for representing photographs in
  computer files.  Efficiency is achieved by discarding parts of the
  visual information which is hardest for the human eye to see.
\item[GIF] An image format well suited for computer generated images
  with few distinct colors, like icons.  Is hampered by a patent on
  the compression scheme used and a maximum of 256 colors.
  Superceeded by PNG.
\item[PNG] The successor to GIF.  Is supported in newer browsers
  only.  
\item[CGI] The Common Gateway Interface.  The original way to generate
  pages and other files dynamically.  A CGI-script can be written in
  any language supported by the web server.

\item[IIS] Microsoft Internet Information Server.  The webserver from
  Microsoft. 
  
\item[ftp] File Transfer Protocol.  This is both the name of the
  \textit{protocol} (the method) as well as the \textit{program}
  implementing the protocol.  ftp can transfer files between a client
  computer and a server computer, and is a standard on the Internet
  
\item[http] Hyper Text Transfer Protocol.  This is the protocol a web
  browser needs to speak with a web server, in order to retrieve
  documents.  Http is intentionally very simple.
\item[TCP/IP] This is the procotol which a computer needs to speak to
  connect to the Internet.  It allows two computer to establish a data
  stream, where one machine pushes bytes in one end of the stream, and
  the other machine retrieves the bytes from the other end of the
  stream, without either computer being concerned about the underlying
  network. 
\item[SP] A \textsf{what}?
\item[Perl] A scripting language which has become popular with Unix
  system administrators, and which ``was there'' when a need for
  scripting languages for CGI arose.  Has eminent regular expressions.
\item[ASP] Microsofts version of a scripting language intertwined with
  HTML. Works on Microsoft web servers only.
\item[JSP] Java Server Pages.  Suns version of Java intertwined with
  HTML.  Requires a servlet-capable web server.
\item[PHP] The Internet version of a scripting language intertwined
  with HTML.  Can run as a CGI script in most browsers, but as a
  module (which is faster) in at least Apache.
\item[Apache] Open Source webserver which runs on a lot of different
  platforms.  Most Linux distributions include it.

\item[PassiveTeX] A package for {\TeX} by Sebastian Rahtz which allows
  FO files to be typeset by {\TeX} directly.

\end{description}

  
\framepage{15cm}{
Explain XML, XSL, XSLT, HTML (yes!), SQL, SAX, DOM, XSP and whatever
fancy terms come to mind.  Draw graph of what happens with an XML
document.}

    
%%% Local Variables: 
%%% mode: latex
%%% TeX-master: "rapport"
%%% End: 
